%%%%%%%
%%%%%%% Conformal Killing Spinor Initial Data 
%%%%%%% 
%%%%%%% arXiv version.
%%%%%%%
%%%%%%%
%%%%%%% Started on: 11.9.2016
%%%%%%% Current version: 31 Jan 2022 
%%%%%%%

\documentclass[10pt,a4paper]{article}
\usepackage{amssymb}
\usepackage{amsmath}
\usepackage{amsfonts}
\usepackage{amsthm}
\usepackage{latexsym}
\usepackage{mathrsfs}
\usepackage{stmaryrd}
\usepackage[dvips]{epsfig}
\usepackage{setspace}
\usepackage{float}
\usepackage{bm}
%\usepackage{showkeys}
%\usepackage[pdftex]{graphicx}
\usepackage{tikz}
\usepackage{enumerate}
\usepackage{subfigure}
\usepackage{nomencl}
%\usepackage{makeidx} 
%\makeindex
\usepackage{authblk}
\renewcommand\Affilfont{\itshape\small}
\usepackage{textcomp}
\usepackage{scrextend}% add KOMA-Script features to other classes
\usepackage[toc,page]{appendix}
\usepackage{comment}
\usepackage{xcolor}
\usepackage[normalem]{ulem}
\newcommand\omicron{o}


\theoremstyle{plain}
\newtheorem{proposition}{Proposition}
\newtheorem{lemma}{Lemma}
\newtheorem{theorem}{Theorem}
\newtheorem{assumption}{Assumption}
\newtheorem*{conjecture}{Conjecture}
\newtheorem*{subconjecture}{Subconjecture}
\newtheorem{corollary}{Corollary}
\newtheorem*{main}{Theorem}
\newtheorem*{definition}{Definition}
\newtheorem{remark}{Remark}

\setlength{\textwidth}{148mm}           % Width of text on page- max 148
\setlength{\textheight}{235mm}          % height of text on page-max 235
\setlength{\topmargin}{-10mm}            % Margin at top ofpage- max -5
\setlength{\oddsidemargin}{0mm}         % Odd page sidemargin max 15
\setlength{\evensidemargin}{0mm}

% Underlined lowcase latin letters
\def\es{{\bar{s}}}
\def\er{{\bar{r}}}

% Boldface mathmode lowcase latin letters
\def\bma{{\bm a}}
\def\bmb{{\bm b}}
\def\bmc{{\bm c}}
\def\bmd{{\bm d}}
\def\bme{{\bm e}}
\def\bmf{{\bm f}}
\def\bmg{{\bm g}}
\def\bmh{{\bm h}}
\def\bmi{{\bm i}}
\def\bmj{{\bm j}}
\def\bmk{{\bm k}}
\def\bml{{\bm l}}
\def\bmn{{\bm n}}
\def\bmm{{\bm m}}
\def\bmo{{\bm o}}
\def\bmq{{\bm q}}
\def\bms{{\bm s}}
\def\bmt{{\bm t}}
\def\bmu{{\bm u}}
\def\bmv{{\bm v}}
\def\bmw{{\bm w}}
\def\bmx{{\bm x}}
\def\bmy{{\bm y}}
\def\bmz{{\bm z}}

% Boldface mathmode numbers
\def\bmzero{{\bm 0}}
\def\bmone{{\bm 1}}
\def\bmtwo{{\bm 2}}
\def\bmthree{{\bm 3}}

% Boldface mathmode uppercase latin letters
\def\bmA{{\bm A}}
\def\bmB{{\bm B}}
\def\bmC{{\bm C}}
\def\bmD{{\bm D}}
\def\bmE{{\bm E}}
\def\bmF{{\bm F}}
\def\bmG{{\bm G}}
\def\bmH{{\bm H}}
\def\bmK{{\bm K}}
\def\bmL{{\bm L}}
\def\bmM{{\bm M}}
\def\bmN{{\bm N}}
\def\bmP{{\bm P}}
\def\bmQ{{\bm Q}}
\def\bmR{{\bm R}}
\def\bmS{{\bm S}}
\def\bmT{{\bm T}}
\def\bmX{{\bm X}}
\def\bmZ{{\bm Z}}


\def\Riem{{\bm R}{\bm i}{\bm e}{\bm m}}
\def\Ric{{\bm R}{\bm i}{\bm c}}
\def\Weyl{{\bm W}{\bm e}{\bm y}{\bm l}}
\def\RWeyl{{\bm R}{\bm W}{\bm e}{\bm y}{\bm l}}
\def\Sch{{\bm S}{\bmc}{\bm h}}
\def\Schouten{{\bm S}{\bmc}{\bm h}{\bm o}{\bm u}{\bm t}{\bm e}{\bm n}}
\def\Hessian{{\bm H}{\bm e}{\bm s}{\bm s}}

% Fracture letters
\def\fraka{{\frak a}}
\def\frakb{{\frak b}}
\def\frakc{{\frak c}}
\def\frakd{{\frak d}}
\def\frakf{{\frak f}}
\def\frakg{{\frak g}}
\def\fraki{{\frak i}}
\def\frakj{{\frak j}}
\def\frakk{{\frak k}}

% Mathbf letters
\def\mbfu{\mathbf{u}}

% Boldface mathmode lowcase greek letters
\def\bmvarkappa{{\bm \varkappa}}
\def\bmbeta{{\bm \beta}}
\def\bmgamma{{\bm \gamma}}
\def\bmdelta{{\bm \delta}}
\def\bmepsilon{{\bm \epsilon}}
\def\bmeta{{\bm \eta}}
\def\bmzeta{{\bm\zeta}}
\def\bmxi{{\bm \xi}}
\def\bmchi{{\bm \chi}}
\def\bmiota{{\bm \iota}}
\def\bmomega{{\bm \omega}}
\def\bmlambda{{\bm \lambda}}
\def\bmmu{{\bm \mu}}
\def\bmnu{{\bm \nu}}
\def\bmphi{{\bm \phi}}
\def\bmvarphi{{\bm \varphi}}
\def\bmsigma{{\bm \sigma}}
\def\bmvarsigma{{\bm \varsigma}}
\def\bmtau{{\bm \tau}}
\def\bmupsilon{{\bm \upsilon}}

% Boldface mathmode uppercase greek letters
\def\bmGamma{{\bm \Gamma}}
\def\bmPhi{{\bm \Phi}}
\def\bmUpsilon{{\bm \Upsilon}}
\def\bmSigma{{\bm \Sigma}}

% Boldface operators
\def\bmpartial{{\bm \partial}}
\def\bmnabla{{\bm \nabla}}
\def\bmhbar{{\bm \hbar}}
\def\bmperp{{\bm \perp}}
\def\bmell{{\bm \ell}}

% Complex and real numbers
\font\SYM=msbm10
\newcommand{\Real}{\mbox{\SYM R}}
\newcommand{\Complex}{\mbox{\SYM C}}
\newcommand{\Natural}{\mbox{\SYM N}}
\newcommand{\Integer}{\mbox{\SYM Z}}
\newcommand{\Sphere}{\mbox{\SYM S}}

% ParallelPerp symbol
\newcommand{\parperp}{\mathbin{\text{\rotatebox[origin=c]{90}{$\models$}}}}
\newcommand{\perppar}{\mathbin{\text{\rotatebox[origin=c]{-90}{$\models$}}}}

%%%%% for piecewise functions right hand side brace %%%%%
\newenvironment{rcases}
  {\left.\begin{aligned}}
  {\end{aligned}\right\rbrace}

% Projector symbols
%\newcommand{\proj2perp}{\pi^{\small{\perp}}}
%\newcommand{\proj2par}{\pi^{\small{\parallel}}}
%\newcommand{\proj4par}{\pi^{\small{\parallel}}}
%\newcommand{\proj4parperp}{\pi^{\small{\parperp}}}
%\newcommand{\proj4perppar}{\pi^{\small{\perppar}}}
%\newcommand{\proj4perp}{\pi^{\small{\perp}}}

%Counter variable for the margin notes
\newcounter{mnotecount}%[section]

% This code generates the margin notes
\newcommand{\mnotex}[1]%{}
{\protect{\stepcounter{mnotecount}}$^{\mbox{\footnotesize $\bullet$\themnotecount}}$ 
\marginpar{%\color{red}%
\raggedright\tiny\em
$\!\!\!\!\!\!\,\bullet$\themnotecount: #1} }

\renewcommand\labelitemi{\tiny$\bullet$}

\newcommand{\notimplies}{%
  \mathrel{{\ooalign{\hidewidth$\not\phantom{=}$\hidewidth\cr$\implies$}}}}


%%%%%%%%%%%%%%%%%%%%%%% needed for long lists of equations
\allowdisplaybreaks
%%%%%%%%%%%%%%%%%%%%%%%%

\begin{document}


\title{\textbf{The conformal Killing spinor initial data equations}}


\author[1]{E. Gasper\'in \footnote{E-mail address:{\tt edgar.gasperin@tecnico.ulisboa.pt}}}
\author[2]{J. L. Williams \footnote{E-mail address:{\tt jlw31@bath.ac.uk}}}
\affil[1]{CENTRA, Departamento de F\'isica,
  Instituto Superior T\'ecnico IST, Universidade de Lisboa UL, Avenida
  Rovisco Pais 1, 1049 Lisboa, Portugal.}
\affil[2]{Department of Mathematical Sciences, University of Bath, Claverton Down, Bath BA2 7AY, United Kingdom.}



\maketitle
\begin{abstract}
  We obtain necessary and sufficient conditions for an initial data
  set for the \emph{vacuum conformal Einstein field equations} to give
  rise to a spacetime development possessing a Killing spinor.  The
  fact that the conformal Einstein field equations are used in our
  derivation allows for the possibility that the initial hypersurface
  $\mathcal{S}$ be (part of) the conformal boundary $\mathscr{I}$.
  For conciseness, these conditions are derived assuming that the
  initial hypersurface is spacelike. Consequently, these equations
  encode necessary and sufficient conditions for the existence of a
  Killing spinor in the development of asymptotic initial data on
  spacelike components of $\mathscr{I}$.
\end{abstract}


\section{Introduction}
\label{sec:Introduction}


   
The discussion of symmetries  in General
Relativity is ubiquitous. From the question of the integrability of the geodesic
equations to the existence of explicit solutions to the Einstein field
equations and the black hole uniqueness problem, symmetries play an important role.   
Symmetry assumptions are usually incorporated into
 the Einstein field equations ---which in vacuum read
\begin{equation}
\tilde{R}_{ab}=\lambda \tilde{g}_{ab},
\label{EFEVacuum}
\end{equation} 
 through the use of Killing vectors.  From the spacetime point of
 view, the existence of Killing vectors allows one to perform
 \emph{symmetry reductions} of the Einstein field equations ---see for
 instance \cite{Wei90a}. This approach has been exploited in classical
 uniqueness results such as \cite{Rob75b}.  Closely related to the
 black hole uniqueness problem, characterisations and classifications
 of solutions to the Einstein field equations usually exploit the
 symmetries of the spacetime in one way or another, e.g. in the
 characterisations of the Kerr spacetime via the \emph{Mars-Simon
 tensor} ---see \cite{Mar99,Mar00,Sim84}.  On the other hand, from the
 point of view of the Cauchy problem, symmetry assumptions should be
 imposed only at the level of initial data. In this regard, symmetry
 assumptions can be phrased in terms of the \emph{Killing vector
 initial data}.  The Killing vector initial data equations constitute
 a set of conditions that an initial data set
 $(\tilde{\mathcal{S}},\tilde{\bmh},\tilde{\bmK})$ ---where
 $\tilde{\mathcal{S}}$ is a 3-dimensional manifold with metric
 $\tilde{\bmh}$ and $\tilde{\bmK}$ denotes the second fundamental
 form--- for the Einstein field equations has to satisfy to ensure
 that the development will contain a Killing vector ---see
 \cite{BeiChr97b}.  Nevertheless, despite the fact that the existence
 of Killing vector plays a central role in the discussion of the
 symmetries, the existence of Killing vectors is sometimes not enough
 to encode all the symmetries and conserved quantities that a
 spacetime can posses, e.g. the Carter constant in the Kerr
 spacetime. To unravel some of these \emph{hidden symmetries} one has
 analyse the existence of a more fundamental type of objects, namely
 \emph{Killing spinors}, denoted here by $\tilde{\kappa}_{AB}$. For vacuum spacetimes,
 the existence of a Killing spinor directly implies the existence of a
 Killing vector. The \emph{Killing spinor initial data equations} have
 been derived in the \emph{physical framework} ---governed by the
 Einstein field equations--- in \cite{GarVal08c}.  These equations
 have been employed in the construction of a geometric invariant which
 detects whether or not an initial data set corresponds to initial
 data for the Kerr spacetime ---see
 \cite{BaeVal10a,BaeVal10b,BaeVal11b}.  This analysis has also been
 extended to include suitable classes of matter ---see \cite{ValCol16}
 for an analogous characterisation of initial data for the Kerr-Newman
 spacetime.  In these characterisations, some asymptotic conditions on
 the initial data are required. These conditions usually take the form
 of decay assumptions on $\tilde{\bmh}$, $\tilde{\bmK}$ and
 $\tilde{\bm\kappa}$ on $\tilde{\mathcal{S}}$, given in terms of
 asymptotically Cartesian coordinates.  Nonetheless, following
 Penrose's proposal, the asymptotic region of the spacetime is to be
 studied in a geometric way through conformal compactifications.  In
 this approach one starts with a \emph{physical spacetime}
 $(\tilde{\mathcal{M}},\tilde{\bmg})$ where $\tilde{\mathcal{M}}$ is a
 4-dimensional manifold and $\tilde{\bmg}$ is a Lorentzian metric
 which is a solution to the Einstein field equations.  Then, one
 introduces a \emph{unphysical spacetime} $(\mathcal{M},\bmg)$ into
 which $(\tilde{\mathcal{M}},\tilde{\bmg})$ is conformally embedded.
Accordingly, there exists an embedding $\varphi: \tilde{\mathcal{M}}
\rightarrow \mathcal{M}$ such that
\begin{equation} \label{eqn:Chapter:Introduction:ConformalRescaling}
\varphi^{*}\bmg=\Xi^2\tilde{\bmg}.
\end{equation}
 By suitably choosing the \emph{conformal factor} $\Xi$ the metric
 $\bmg$ may be well defined at the points where $\Xi=0$. In such
 cases, the set of points for where the conformal factor vanishes is
 at infinity from the physical spacetime perspective.
\noindent The set
\[
 \mathscr{I} := \big\{p \in \mathcal{M} \hspace{0.2cm}| \hspace{0.2cm} \Xi(p)=0
 , \hspace{0.2cm} \mathbf{d}\Xi(p) \neq0\big\}
\]
is called the conformal boundary.  However, it can be readily
verified that the Einstein field equations are not conformally
invariant. Moreover, a direct computation using the conformal
transformation formula for the Ricci tensor shows that the vacuum
Einstein field equations \eqref{EFEVacuum} lead to an equation which
is formally singular at the conformal boundary.  An approach to deal
with this problem was given in \cite{Fri81a} where a regular set of
equations for the unphysical metric was derived. These equations are
known as the \emph{conformal Einstein field equations} (CFEs).  The crucial
property of these equations is that they are regular at the points
where $\Xi=0$ and a solution thereof implies whenever $\Xi\neq 0$ a
solution to the Einstein field equations ---see \cite{Fri81a,Fri83}
and \cite{CFEbook} for a comprehensive discussion.  There are three
ways in which these equations can be presented: the metric, the frame
and spinorial formulations. These equations have been mainly used in
the stability analysis of spacetimes ---see for instance \cite{Fri86b,
  Fri86c} for the proof of the global and semi-global non-linear
stability of the de Sitter and Minkowski spacetimes, respectively.
Expressed in a nutshell, the main advantage of the conformal
(unphysical) approach to the Einstein field equations is that the
conformal boundary $\mathscr{I}$, determined by $\Xi=0$, is a
submanifold of $(\mathcal{M},\bmg)$. This allows, in particular, to
consider the $\Xi=0$ hypersurface as a legitimate hypersurface to
prescribe data which can be evolved using regular ---without
$\Xi^{-1}$-terms--- evolution equations. 
This set up is particularly attractive to study
spacetimes with $\lambda>0$ in which, given the appropriate
conditions, the conformal boundary is a spacelike
hypersurface and hence one can pose \emph{an asymptotic initial value
problem}: an initial value problem where the initial hypersurface is
$\mathscr{I}$ ---see \cite{GasVal17,MarPaeSenSim16, LueVal09}.

\medskip

The conformal counterpart of the Killing vector initial data equations
\cite{BeiChr97b} was derived in \cite{Pae14a}.  In the latter
reference, intrinsic conditions on an initial hypersurface
$\mathcal{S}\subset \mathcal{M}$ of the unphysical spacetime are found
such that the development of the data gives rise to a conformal
Killing vector of the unphysical spacetime $(\mathcal{M},\bmg)$ which,
in turn, corresponds to a Killing vector of the physical spacetime
$(\tilde{\mathcal{M}},\tilde{\bmg})$.  Notice that this construction
allows $\mathcal{S}$ to be determined by $\Xi=0$ so that it to
corresponds to $\mathscr{I}$.  To avoid confusion, we emphasise that
the equations of \cite{Pae14a} are derived in the \emph{unphysical
framework} ---governed by the CFEs---
and are not the same as those derived in \cite{GarKha19} which
represent restrictions on the initial data for the physical spacetime
$(\tilde{\mathcal{M}},\tilde{\bmg})$ to give rise to a conformal
Killing vector.
%%%%%%%%%%%%%%%%%%%%%%%%
%\mnotex{I added this reference because Alfonso
% could be one of the referees of our paper}
%%%%%%%%%%%%%%%%%%%%%%%%
On the other hand, as previously mentioned, in the case of Petrov type
D spacetimes such as the Kerr-de Sitter spacetime, the symmetries of
the spacetime are closely related to the existence of Killing spinors.
Hence, a natural question in this setting is whether a conformal
counterpart ---in the unphysical framework--- of the Killing spinor
initial data equations introduced in \cite{GarVal08c} can be
obtained. In other words, what are the extra conditions that one has
to impose on an initial data set for the CFEs so that the arising
development contains a Killing spinor?  This question is answered in
this article by deriving such conditions which we call the
\emph{conformal Killing spinor initial data equations}.

\medskip

Although the Killing spinor equation is conformally invariant, it is
not a priori clear whether the conditions of \cite{GarVal08c,
  BaeVal10b} may be translated directly into the unphysical setting.
Indeed, one expects this not to be the case, since the Einstein field
equations are not conformally invariant.  Although the results of
\cite{GarVal08c} may be recovered from the analysis presented here by
setting $\Xi = 1$, an important difference is that the set of
variables that allow to obtain a closed system of homogeneous wave
equations in the present case are different. The need for a different
set of \emph{Killing spinor zero-quantities} propagated in the
conformal case, can be traced back to the observation that in
$(\mathcal{M},\bmg)$ the vector
$\xi_{AA'}=\nabla_{A'}{}^{Q}\kappa_{QA}$ does not correspond to a
(conformal) Killing vector. Although for a general Lorentzian manifold
this vector appears not to have any clear geometric significance, a
by-product of the present analysis is that, for conformally Einstein
manifolds ---namely, solutions to the CFEs---
the vector $\xi_{AA'}$ represents a \emph{Weyl
collineation} ---see \cite{KatLevDav69} for definitions of curvature
collineations.  Once the existence of a Killing spinor is established
one can use the conformal factor $\Xi$, the Killing spinor
$\kappa_{AB}$ and $\xi_{AA'}$, to construct a conformal Killing vector
$X_{a}$ associated to a Killing vector $\tilde{X}_{a}$ of the physical
spacetime $(\tilde{\mathcal{M}},\tilde{\bmg})$. In the analysis of
\cite{GarVal08c} the fact that
$\tilde{\xi}_{AA'}=\tilde{\nabla}_{A'}{}^{Q}\tilde{\kappa}_{QA}$ is a
Killing vector is crucial and one propagates off the initial
hypersurface, simultaneously, $\tilde{\kappa}_{AA'}$ and
$\tilde{\xi}^{AA'}$.  The latter motivates the introduction of
$\tilde{S}_{ab} := \tilde{\nabla}_{(a}\tilde{\xi}_{b)}$ as a
zero-quantity in the physical framework of \cite{GarVal08c}.  Similarly,
in the work of \cite{ValCol16} where the results of \cite{GarVal08c}
are generalised to the case where $(\tilde{\mathcal{M}},\tilde{\bmg})$
satisfies the Einstein-Maxwell equations, the condition
$\tilde{S}_{ab}=0$ is verified by virtue of the so-called \emph{matter
alignment condition}.  In the conformal setting analysed in this
article, the analogous quantity $S_{ab}$ is not as geometrically
motivated as in the physical cases and its usage as a variable in the
system does not lead to a closed system of explicitly regular
homogeneous wave equations. Here the adjective regular refers to the
absence of formally singular terms, such as $\Xi^{-1}$, in the
equations. Instead, the variable that is central for the present
analysis turns out to be the so-called \emph{Buchdahl constraint} (and
derivatives thereof), which relates the existence of Killing spinors
with the Petrov type of $(\mathcal{M},\bmg)$.


\medskip
   The core of the proof of the main theorem of this article is
   obtaining a closed system of homogeneous wave equations for fields
   encoding the existence of a Killing spinors. Although these wave
   equations hold regardless of the causal character of $\mathscr{I}$,
   to obtain intrinsic conditions to $\mathcal{S}$ we assume, for conciseness
   that $\mathcal{S}$ is spacelike.  Nonetheless, a similar
   computation can be performed on an hypersurface $\mathcal{S}$ with
   a different causal character. Therefore, the conditions found in this article
   have potential applications for the black hole uniqueness problem.
   In particular, they can be used for an asymptotic characterisation
   of the Kerr-de Sitter spacetime analogous to \cite{MarPaeSenSim16}
   in terms of the existence of Killing spinors at the conformal
   boundary $\mathscr{I}$.

\medskip

The main result of this article ---the more precise statement of which
can be found in Theorem \ref{Theorem_KS}--- is summarised informally
in the following:

\begin{main}\label{TheoremSummary}
If the conformal Killing spinor initial data equations
 \eqref{CSKID_1}-\eqref{CSKID_2}  are satisfied
on an open set $\mathcal{U}\subset \mathcal{S}$, where
 $\mathcal{S}$ is a spacelike hypersurface on which initial data for 
the conformal Einstein field equations has been prescribed,
 then, the domain of dependence of $\mathcal{U}$  possesses a Killing spinor.
\end{main}

    Although the main objective of the present paper is deriving the
    valence-2 Killing spinor initial data in the conformal setting
    $(\mathcal{M},\bmg)$, the analogous conditions encoding the
    existence of a valence-1 Killing spinor are also derived.  The
    latter serves as a warm-up exercise for the valence-2 case where
    one can already observe the above discussed features and
    understand differences between the derivation of the conditions on
    $(\tilde{\mathcal{M}},\tilde{\bmg})$ and those on
    $(\mathcal{M},\bmg)$ in a simpler arena.

\medskip


Involved computations throughout this article were facilitated through
the suite {\tt xAct} in {\tt Mathematica}.



\subsection*{Overview of the article}
  Section \ref{Background} summarises relevant background material:
  subsection \ref{NotationAndSpinorFormalism} fixes the
conventions and notation and 
gives an abridged discussion of the main spinorial identities to be
used and the space spinor formalism;  subsection \ref{Sec:KillingSpinors}
gives an overview of Killing spinors and their conformal
properties; subsection \ref{Sec:CFEs}  the conformal
Einstein field equations are presented for later use.  In Section
\ref{conformalTwistorKID} the conformal (valence-1 Killing spinor)
twistor initial data equations are obtained. In Section
\ref{conformalKSKID} the conformal (valence-2) Killing initial data equations are
derived and discussed.


\subsection*{Notations and conventions}

Throughout this article, $(\mathcal{M}, \bmg)$ will denote a
4-dimensional manifold equipped with a Lorentzian metric $\bmg$ of
signature $(+, -, -, -)$, with associated Levi-Civita connection
$\tilde{\nabla}$.   We moreover assume $(\mathcal{M}, \bmg)$ to be
globally-hyperbolic. The Upper case Latin indices ~$_{ABC\cdots A'B'C'}$~
will be used as abstract indices of the \emph{spacetime spinor}
algebra, and the bold numerals ~$_{\bm0\bm1\bm2\cdots}$~ denote
components with respect to a fixed spin dyad $ o^A:=
\epsilon_{\bm0}{}^A,\iota^A:=\epsilon_{\bm1}{}^A $ ---see Penrose
\& Rindler \cite{PenRin84} for further details.
Lower case Latin indices $_{a,b,c...}$ will be used as abstract tensor
indices.  For tensors, our curvature conventions are fixed by
\[\nabla_{a}\nabla_{b}\kappa^c-\nabla_{b}\nabla_{a}\kappa^c=R_{ab}{}^{c}{}_{d}\kappa^{d}.\]
The future domain of dependence of an achronal set $\mathcal{A}$ will be denoted
by $\mathcal{D}^{+}(\mathcal{A})$.


\section{Background}
\label{Background}

In this section, we give an abridged recap of spacetime and space spinor
calculus, in addition to giving a brief introduction to Killing
spinors and the conformal Einstein field equations.

\subsection{Spinorial formalism in a nutshell}
\label{NotationAndSpinorFormalism}

Since $(\mathcal{M}, \bmg)$ is, by assumption, globally-hyperbolic, it
admits a spinor structure ---see Proposition $4$ in
\cite{CFEbook}. The use of the spinor structure of $(\mathcal{M},
\bmg)$ fixes the signature to $(+, -, -, -)$.

\medskip

For spinors, the curvature conventions are fixed via the spinorial
Ricci identities which will be written in accordance with the above
convention for tensors.  To see this, recall that the commutator of
covariant derivatives $[ \nabla_{AA'},\nabla_{BB'}]$ can be expressed
in terms of the symmetric operator $\square_{AB}$ as
\[
[ \nabla_{AA'},\nabla_{BB'}]= \epsilon_{AB}\square_{A'B'} +
\epsilon_{A'B'}\square_{AB}
\]
where
\[
\square_{AB} := \nabla_{Q'(A} \nabla_{B)}{}^{Q'}.
\]
 The action of the symmetric operator $\square_{AB}$ on valence-1
 spinors is encoded in the spinorial Ricci identities
\begin{subequations}
\begin{eqnarray}
&& \square_{AB}\xi_{C}=-\Psi_{ABCD} \xi^{D} +
  2\Lambda\xi_{(A}\epsilon_{B)C},
 \label{SpinorialRicciIdentities1} \\
&& \square_{A'B'}\xi_{C}=-\xi^{A}\Phi_{CA A' B'},
\label{SpinorialRicciIdentities2}
\end{eqnarray}
\end{subequations}
where $\Psi_{ABCD}$ and $\Phi_{AA'BB'}$ and $\Lambda$ are curvature
spinors.  The above identities can be extended to higher valence
spinors in the obvious way ---further discussion (albeit using
slightly different conventions) can be found in \cite{Ste91}. A
related identity which will be systematically used in the following
discussion is
\begin{equation}\label{DecomposeDoubleDerivativeContracted}
\nabla_{AQ'}\nabla_{B}{}^{Q'}=\square_{AB}+
\tfrac{1}{2}\epsilon_{AB}\square,
\end{equation}
where $\square_{AB}$ is the symmetric operator defined above and
$\square := \nabla_{AA'}\nabla^{AA'}.$
\medskip 

To have a self-contained discussion the space spinor
formalism, originally introduced in \cite{Som80}, is briefly recalled
---see also \cite{GarVal08c,BaeVal10b,CFEbook}.  Let $\tau^{AA'}$
denote the spinorial counterpart of a timelike vector $\tau^{a}$,
normal to a spacelike hypersurface $\mathcal{S}$ and normalised so
that $\tau_{a}\tau^{a}=2$.  Then, it follows that
$\tau_{AA'}\tau^{AA'}=2$ and, consequently,
\[\tau_{AA'}\tau_B{}^{A'}=\epsilon_{AB}.\]
Given a spacetime spinor $u_{AA'}$, its space spinor decomposition
reads
\[
u_{AA'}= \tfrac{1}{2}\tau_{AA'}u-\tau^{B}{}_{A'}u_{(AB)},
\]
where $u:=\tau^{AA'}u_{AA'}$ and $u_{(AB)}:=\tau_{(A}{}^{B'}u_{B)B'}$.
This split extends to higher valence spinors in an analogous way
---see \cite{GarVal08c,BaeVal10b,CFEbook}. Similarly,
the covariant derivative $\nabla_{AA'}$ is then decomposed into the
\emph{normal} and \emph{Sen} derivatives:
\begin{align*}
  %& \mathcal{P}
  \nabla_{\bm\tau} := \tau^{AA'}\nabla_{AA'},\qquad  \mathcal{D}_{AB}:=
  \tau_{(A}{}^{A'}\nabla_{B)A'}.
\end{align*}
Though we will not need them here, for completeness we note that the \emph{Weingarten} spinor and the \emph{acceleration} of the
congruence are then defined by
\[K_{ABCD} := \tau_{D}{}^{C'} \mathcal{D}_{AB}\tau_{CC'},\qquad K_{AB} := \tau_{B}{}^{C'} \nabla_{\bm\tau}\tau_{AC'}.
\]
%%%%%%%%%%%%%%%%%%%%%%%%%%%%%%%%%%%%%%%%%%%%%%%%%%%%%%%%%
%%\mnotex{these are not really used so
%% I cutted some of the discussion to make it more brief}
%% The above can be inverted to obtain the following formulae which will
%% prove useful in the sequel
%% \begin{align*}
%%   & \nabla_{\bm\tau} \tau_{CC'}=- K_{CD} \tau^{D}{}_{C'},\\ &
%%   \mathcal{D}_{AB}\tau_{CA'} = - K_{ABCD} \tau^{D}{}_{A'}.
%% \end{align*}
%%%%%%%%%%%%%%%%%%%%%%%%%%%%%%%%%%%%%%%%%%%%%%%%%%%%%%%%%
The distribution induced by $\tau_{AA'}$ is integrable if and only
$K^D{}_{(AB)D}=0$, in which case $K_{ABCD}$ describes the extrinsic
curvature of the resulting foliation. Nevertheless, this is not required for our subsequent discussion; we will allow
 for the possibility that the distribution is non-integrable.\mnotex{Is this true? Do we really not need an integrable manifold.}
%---i.e. the spinor $ K^D{}_{(AB)D}$ will not be assumed
%to vanish.
%%%%%%%%%%%%%%%%%%%%%%%%%%%%%%%%%%%%%%%%%%%%%%%%%%%%%%%%%%
%%\mnotex{these are not really used so
%% I cutted some of the discussion to make it more brief}
%% \medskip
%%
%% Defining the spinors $\chi_{AB}:= K^D{}_{(AB)D}$,
%% $\chi_{ABCD}:= K_{(ABCD)}$ and $\chi:= K_{AB}{}^{AB}$, the
%% Weingarten spinor decomposes as follows
%% \begin{equation}
%% \label{ExtrinsicCurvatureSplit}
%% K_{ABCD} = \chi_{ABCD} - \tfrac{1}{2} \epsilon_{A(C}\chi_{D)B} -
%% \tfrac{1}{2} \epsilon_{B(C}\chi_{D)A} - \tfrac{1}{3} \chi
%% \epsilon_{A(C} \epsilon_{D)B}.
%% \end{equation}
%%%%%%%%%%%%%%%%%%%%%%%%%%%%%%%%%%%%%%%%%%%%%%%%%%%%




\subsection{Killing spinors}\label{Sec:KillingSpinors}

To start the discussion it is convenient to introduce some notation
and definitions. Let $(\tilde{\mathcal{M}},\tilde{\bmg})$ be a
4-dimensional manifold equipped with a Lorentzian metric
$\tilde{\bmg}$ and denote by $\tilde{\nabla}$ its associated
Levi-Civita connection.  Later, we will reserve the $\tilde{\cdot}$
notation for a vacuum spacetime ---that is to say, a solution of the
vacuum Einstein field equations \eqref{EFEVacuum}. We note however
that, in the present section, no such restriction is necessary.
\medskip

A totally symmetric
$\tilde{\kappa}_{A_1...A_p}=\tilde{\kappa}_{(A_1...A_p)}$ valence$-p$
spinor is said to be a (valence$-p$) \emph{Killing spinor} if is
satisfies the following equation
\begin{equation}\label{qValenceKillingspinor}
\tilde{\nabla}_{Q'(Q}\tilde{\kappa}_{A_1...A_p)}=0.
\end{equation}
An important property of the Killing spinor equation is that it is
conformally-invariant, in other words if $\bmg$ is conformally related
to $\tilde{\bmg}$ ---namely $\bmg=\Xi^2\tilde{\bmg}$--- then
${\kappa}_{A_1...A_q}=\Xi^2 \tilde{\kappa}_{A_1...A_q}$ satisfies
\[{\nabla}_{Q'(Q}{\kappa}_{A_1...A_p)}=0.\]
\medskip
\noindent In this paper we will only focus only the case $p=1$ and
$p=2$.  If $p=1$, the equation
\begin{equation}\label{TwistorEq}
  \tilde{\nabla}_{Q'(Q}\tilde{\kappa}_{A)}=0.
\end{equation}
is usually referred as the \emph{twistor equation}, and a solution
referred to as a \emph{twistor}; we will follow this naming convention
here.  The valence-2 case, on the other hand, will be referred to
simply as the \emph{Killing spinor case}.  Namely, we will say that a
symmetric valence$-2$ spinor,
$\tilde{\kappa}_{AB}=\tilde{\kappa}_{(AB)}$, is a \textit{Killing
  spinor} if it satisfies the equation
\begin{equation}
\tilde{\nabla}_{A'(A}\tilde{\kappa}_{BC)}=0.
\end{equation}
The Killing spinor equation and twistor equations are, in general,
overdetermined; in particular, they imply the so-called
\textit{Buchdahl constraint}.  In the twistor case ($p=1$) this has
the form
\[
\tilde{\kappa}^D\Psi_{ABCD}=0,
\]
while in the Killing spinor case ($p=2$) the Buchdahl constraint
takes the form
\[
\tilde{\kappa}_{(A}{}^Q\Psi_{BCD)Q}=0,
\]
where $\Psi_{ABCD}$ denotes the conformally invariant Weyl spinor.
The latter condition restricts $\Psi_{ABCD}$ to be algebraically
special.  In the twistor case the spacetime is necessarily of Petrov
type N or O, hence restricting its applicability for characterisation
of black holes.  In the Killing spinor case the spacetime is only
restricted to be of Petrov type D, N or O. 
%In the type N case, choosing an 
%adapted dyad $\lbrace \bmo, \bm\iota\rbrace$ for which $\psi_{ABCD}=o_Ao_Bo_Co_D$, 
%a Killing twistor and spinor are given explicitly 
%by 
%\[\tilde{\kappa}_A=o_A, \qquad\text{and}\qquad  \tilde{\kappa}_{AB}=o_Ao_B.\]
In the type D case, choosing an 
adapted dyad $\lbrace \bmo, \bm\iota\rbrace$ for which $\psi_{ABCD}=\psi o_{(A}o_{B}\iota_C\iota_{D)}$
\[\tilde{\kappa}_{AB} = \psi^{-1/3}o_{(A}\iota_{B)}\]
is a Killing spinor. Indeed, the fact that the Killing spinor equation is satisfied follows from the Bianchi identity $\tilde{\nabla}^A{}_{A'}\Psi_{ABCD}=0$ ---see \cite{PenRin84, WalkerPenrose70} for more details. 
\medskip

At first glance, the conformal invariance property of the Killing
spinor equation would seem to indicate that the approach leading to
the Killing spinor initial data conditions derived in \cite{GarVal08c}
would identically apply for $(\mathcal{M},\bmg)$ with
$\tilde\bmg=\Xi^2\tilde{\bmg}$. This is not the case simply because
the Einstein field equations are not conformally invariant.  In other
words, in the analysis of \cite{GarVal08c} the vacuum Einstein field
equations \eqref{EFEVacuum} were used, and, despite that one can
relate $R_{ab}$ with $\tilde{R}_{ab}$ this leads to formally singular
terms (terms containing $\Xi^{-1}$). Moreover, even if one is willing
to work with formally singular equations it is not a priori clear that
the choice of variables made in \cite{GarVal08c} will form a closed
homogeneous system in the conformal setting.  To see why this is the
case, notice that for general manifold with metric
$(\tilde{\mathcal{M}},\tilde{\bmg})$ ---namely $\tilde{\bmg}$ not
satifying any field equation--- the existence of a Killing spinor
$\tilde{\kappa}_{AB}$ is not related to the existence of a
Killing vector.
Nevertheless, if one assumes that $\tilde{\bmg}$
satisfies the vacuum Einstein field equations \eqref{EFEVacuum} then
the concomitant
\begin{equation*}
\tilde{\xi}_{AA'} := \tilde{\nabla}^{B}{}_{A'}\tilde{\kappa}_{AB},
\end{equation*}
represents the spinorial counterpart of a complex Killing vector of
the spacetime $(\tilde{\mathcal{M}},\tilde{\bmg})$ ---see
\cite{GarVal08c} for further discussion. This point is subtle and even
in the physical (non-conformal) framework if one is to include matter
such as the Maxwell field and the analysis of \cite{GarVal08c} does
not straighfowardly apply since further conditions (the matter
alignment conditions) ---see \cite{ValCol16}--- need to be propagated.
\begin{remark}
  \emph{
  The notion of Killing spinors is related to that
  of Killing--Yano tensors. If a Killing spinor
$\tilde{\xi}_{AA'}$ is Hermitian, i.e.,
$\bar{\tilde{\xi}}_{AA'}=\tilde{\xi}_{AA'}$, then one can construct the
spinorial counterpart of a \emph{Killing--Yano tensor}
$\tilde{\Upsilon}_{ab}$ ---i.e. an antisymmetric $2-$tensor satisfying
$\tilde{\nabla}_{(a}\tilde{\Upsilon}_{b)c}=0$--- as follows
\[\tilde{\Upsilon}_{AA'BB'}=i(\tilde{\kappa}_{AB}\bar{\tilde{\epsilon}}_{A'B'}
-\bar{\tilde{\kappa}}_{A'B'}\tilde{\epsilon}_{AB}).\] Conversely,
given a Killing--Yano tensor, one can construct a Killing spinor
---see \cite{ValCol16,McLBer93,PenRin86}.}
\end{remark}
\medskip

From now on, $(\tilde{\mathcal{M}},\tilde{\bmg})$ will be reserved to
denote the \emph{physical spacetime}, in other words, the symbol
$\tilde{ \quad}$ will be added to those fields associated with a
solution $\tilde{\bmg}$ to the vacuum Einstein field equations
\eqref{EFEVacuum}.  Similarly $(\mathcal{M},\bmg)$ will be used to
represent the \emph{unphysical spacetime} related to
$(\tilde{\mathcal{M}},\tilde{\bmg})$ via $\bmg=\Xi^2\tilde{\bmg}$
---as customary, in a slight abuse of notation, the pullback $\varphi^*$ of the
 embedding
%%%%%%%%%%%%%%%
%$\varphi(\tilde{\mathcal{M}})$ and
%$\mathcal{M}$ will be identified so that
%the mapping
%%%%%%%%%%%%%%
$\varphi:
\tilde{\mathcal{M}}\rightarrow\mathcal{M}$ will be omitted.



\subsection{The  conformal Einstein field equations}
\label{Sec:CFEs}

This section contains an abriged discussion of the conformal Einstein
field equations.  At the end of this section the main technical
tool from the theory of partial
differential equations to be used for deriving the Killing spinor
intial data equations is given.

\medskip
 

The conformal Einstein field equations (CFEs) are a conformal
formulation of the Einstein field equations. In other words, given a
spacetime $(\tilde{\mathcal{M}},\tilde{\bmg})$ satisfying the Einstein
field equations, the CFEs
%%%%%%%%
%conformal Einstein field equations
%%%%%%%
encode a system of differential conditions for the curvature and
concomitants of the conformal factor associated with
$(\mathcal{M},\bmg)$ where $\bmg=\Xi^2\tilde{\bmg}$. The key property
of these equations is that they are regular even at the conformal
boundary $\mathscr{I}$, where $\Xi=0$.  The CFEs were derived originally
in \cite{Fri81a} ---see also \cite{CFEbook} for a
comprehensive discussion.
%%%%%%%%%%%%%%%
%% This formulation of the CFEs  conformal Einstein field equations
%% was first given in \cite{Fri81a} ---see also \cite{CFEbook} for a
%% comprehensive discussion.
%%%%%%%%%%%%%%%

\medskip

The metric version of the standard vacuum conformal 
Einstein field equations are encoded in the following zero-quantities
  ---see \cite{Fri81a,Fri81b,Fri82,Fri83}:
\begin{subequations}\label{CFE_tensor_zeroquants}
\begin{eqnarray}
&& Z_{ab} := \nabla_{a}\nabla_{b}\Xi  +\Xi L_{ab} - s g_{ab}=0 ,
 \label{StandardCEFEsecondderivativeCF}\\
&& Z_{a} := \nabla_{a}s +L_{ac} \nabla ^{c}\Xi=0 , \label{standardCEFEs}\\
&& \delta_{bac} := \nabla_{b}L_{ac}-\nabla_{a}L_{bc} -
 d_{abcd}\nabla^d{}\Xi =0 , \label{standardCEFESchouten}\\
&& \lambda_{abc}:= \nabla_{e}d_{abc}{}^{e}=0 , \label{standardCEFErescaledWeyl}\\
&& Z := \lambda - 6 \Xi s + 3 (\nabla_{a}\Xi) \nabla^{a}\Xi,
\label{standardCFEconstraintFriedrichScalar}
\end{eqnarray}
\end{subequations}
where $\Xi$ is the conformal factor, $L_{ab}$ is the Schouten tensor,
defined in terms of the Ricci tensor $R_{ab}$ and the Ricci scalar $R$
via
\begin{equation}\label{SchoutenDefinition}
L_{ab}=\tfrac{1}{2}R_{ab}-\tfrac{1}{12}Rg_{ab},
\end{equation}
 $s$ is the so-called \emph{Friedrich scalar} defined as
\begin{equation}\label{s-definition}
s:= \tfrac{1}{4}\nabla_{a}\nabla^{a}\Xi + \tfrac{1}{24}R\Xi,
\end{equation}
and $d^{a}{}_{bcd}$ denotes the \emph{rescaled Weyl tensor}, defined
as
\[d^{a}{}_{bcd}=\Xi^{-1}C^{a}{}_{bcd},\]
where $C^{a}{}_{bcd}$ denotes the Weyl tensor.  The geometric meaning
of these zero-quantities is as follows. The equation $Z_{ab}=0$
encodes the conformal transformation law between ${R}_{ab}$ and
$\tilde{R}_{ab}$.  The equation $Z_{a}=0$ is obtained considering
$\nabla^{a}Z_{ab}$ and commuting covariant derivatives.  Equations
$\delta_{abc}=0$ and $\lambda_{abc}=0$ encode the contracted second
Bianchi identity. Finally, $Z=0$ is a constraint in the sense that if
it is verified at one point $p\in\mathcal{M}$ then $Z=0$ holds in
$\mathcal{M}$ by virtue of the previous equations.  A solution to the
metric conformal Einstein field equations consists of a collection of
fields
\[
\{g_{ab}, \; \Xi, \; \nabla_{a}\Xi,s\;,L_{ab},\; d_{abcd}\}
\]
satisfying
\begin{equation}%\label{vanishing_CFEs_tensorial_zq}
  Z_{ab}=0, \quad Z_{a}=0, \quad \delta_{abc}=0, \quad \lambda_{abc}=0, \quad Z=0.
\end{equation}

%% \begin{remark}
%% \emph{ In the metric formulation of the CFEs
%%   equations one needs to supplement the system encoded in the zero
%%   quantities defined above with an equation for the unphysical metric
%%   $g_{ab}$. To do so, one considers equation
%%   \eqref{SchoutenDefinition} expressed in some local coordinates
%%   $(x^{\mu})$. Recalling that in local coordinates the components of
%%   the Ricci tensor can be written as second order derivatives of the
%%   metric, one obtains the required equation for the unphysical metric.
%%   A hyperbolic reduction can be obtained through the use of
%%   gauge source functions or other hyperbolic reduction stratagies
%%   used in standard formulations of the Einstein field equations.
%% }
%% \end{remark}


 \begin{remark}
   \emph{ If one opts to use the Ricci tensor $R_{ab}$ instead of the
     Schouten tensor $L_{ab}$ then the Ricci scalar $R$ appears in the
     right-hand side of equations but no equation for it has been
     provided.  In the CFEs the Ricci scalar encodes the
     \emph{conformal gauge source function}, hence there is no
     equation to fix that variable as it represents a gauge quantity
     of the formulation.}
 \end{remark}

\noindent Since we are concerned here with spinor fields, we will need the
spinorial transcription of the CFEs, which reads as follows
\begin{subequations}
\begin{eqnarray}
   &&  Z_{AA'BB'}  =  \nabla_{AA'}\nabla_{BB'}\Xi - \Xi \Phi _{ABA'B'}  - s \epsilon _{AB} \epsilon
  _{A'B'} + \Xi \Lambda \epsilon _{AB} \epsilon _{A'B'} ,
  \label{Def_ConfFactor_CFE_zeroquant}\\
  && Z_{AA'}  = \nabla_{AA'}s + \Lambda  \nabla_{AA'}\Xi   - \Phi _{ABA'B'} \nabla^{BB'}\Xi ,\label{Def_s_CFE_zeroquant}\\
 &&  \delta_{ABCC'} =   \nabla_{A'(A}\Phi _{B)CC'}{}^{A'} - \epsilon _{C(A} \nabla_{B)C'}\Lambda   +  \phi _{ABCD} \nabla^{D}{}_{C'}\Xi ,\label{Def_delta_CFE_zeroquant} \\
  && \Lambda _{C'ABC}  = \nabla_{DC'}\phi _{ABC}{}^{D}, \label{Def_Lambda_CFE_zeroquant}\\
   && Z  = \lambda   -6 \Xi  s + 3 (\nabla_{AA'}\Xi)  \nabla^{AA'}\Xi. \label{Def_cons_CFE_zeroquant}
\end{eqnarray}
\end{subequations}
See \cite{CFEbook} for further details.
As in the tensorial case, one can choose the
Schouten (tensor) spinor or the Ricci (tensor) spinor as a variable.
Here the equations have been expressed
using the standard curvature spinors of the NP formalism,
namely, the trace-free Ricci spinor $\Phi_{ABA'B'}$, the Ricci scalar
$\Lambda$ ---in fact $R= 24\Lambda$--- and the Weyl spinor
$\Psi_{ABCD}$ which enters via the \emph{rescaled Weyl spinor}, $\phi_{ABCD}$,
defined as 
\begin{equation}\label{Def_rescaled_Weyl_spinor}
\phi_{ABCD} := \Xi^{-1} \Psi_{ABCD}
\end{equation}
 ---see \cite{Ste91, PenRin84} for more details. 
 \begin{remark}
 \label{G-C-M-Remark}
   \emph{The initial data for $\phi_{ABCD}$ is determined using the Gauss-Codazzi and Codazzi-Mainardi equations. First decompose 
   \[ \phi_{ABCD}=E_{ABCD}+iB_{ABCD}\]
where $\bmE$ and $\bmB$ are the \textit{electric} and \textit{magnetic} parts
   \[
   E_{ABCD} := \tfrac{1}{2}(\phi_{ABCD} + \hat{\phi}_{ABCD}),\qquad B_{ABCD} := \tfrac{i}{2}(-\phi_{ABCD} + \hat{\phi}_{ABCD}),
  \]
  with $\hat{\phi}_{ABCD}:=\tau_A{}^{A'}\tau_B{}^{B'}\tau_C{}^{C'}\tau_D{}^{D'}\bar{\phi}_{A'B'C'D'}$. The fields $\bmE,\bmB$ are the spinorial counterparts of the electric and magnetic parts of the rescaled Weyl tensor and comprise initial data: away from $\mathscr{I}$ they are determined by the Gauss--Codazzi--Mainardi equations, while
   for the asymptotic initial value problem, the constraint equations
   implied by the CFEs acquire a particularly simple form so that
   initial data for the magnetic part is
   determined algebraically by 
   %%%%
   %%$y_{ij}$,
   %%%
   the Bach tensor of $h_{ij}$ and the
   electric part is prescribed, the only constraint being that it satisfies the \textit{TT} condition with respect to $h_{ij}$---see
   \cite{CFEbook, GasVal17a}.
   For the discussion of this paper we will
   assume that such data $\bmE, \bmB$ has been already determined.}
\end{remark}
The CFEs as previously presented can be regarded as a set of covariant
conditions for geometric fields on $(\mathcal{M},\bmg)$ and, hence,
they do not have a particular PDE character.  However, there are,
depending on the gauge fixing procedure, different hyperbolic
reduction strategies to extract a set of evolution and constraint
equations.  For the subsequent discussion only the evolution and
constraint equations implied by the $\Lambda_{C'ABC}=0$ equation, namely,
\begin{equation}\label{RescaledWeylEquationDisplayed}
 \nabla^D{}_{C'}\phi _{ABCD}=0,
\end{equation}
%%%% \mnotex{Equation displayed explicitly for making
%%%% reference to it in a succint way in the text in section 4}
will play an important role.  A direct calculation using
the space spinor formalism
shows that equation \eqref{RescaledWeylEquationDisplayed}
%%$\Lambda_{CC'AB}=0$
can be recast as the following system
of evolution and constraint equations
\begin{align}\label{RescaledWeyl_evo_const}
%%\nabla_{\bm\tau} \phi _{ABCD} = -2 \mathcal{D} _{DF}\phi _{ABC}{}^{F}
  & \nabla_{\bm\tau} \phi _{ABCD} = 2 \mathcal{D} _{(A}{}^{F}\phi_{BCD)F},
%% \label{RescaledWeyl_evo}\\
%%& \qquad
  \qquad \mathcal{D} ^{CD}\phi _{CDAB} = 0.
 %%\label{RescaledWeyl_const}
\end{align}
The evolution and constraint equations associated to the other
zero-quantities depend on the particular gauge fixing strategy and
will not play a relevant role for the discussion in the forthcoming
sections.
\medskip

The CFEs are usually presented as the first-order system
\eqref{vanishing_CFEs_tensorial_zq} with the definitions
\eqref{CFE_tensor_zeroquants}. However, for several applications it is
convenient to use a second-order formulation of the equations. In
\cite{Pae13} the tensorial version of the CFEs was recast as a set of
(tensorial) wave equations.  Analogously, in \cite{GasVal15} a
second-order form of the spinorial formulation of the CFEs ---the
\emph{spinorial CFE wave equations} ---was obtained.  This version of
the CFEs is particularly suited for the applications of this article;
though, in fact, only one of those equations ---that for the rescaled
Weyl spinor--- will be needed.
%%%%%%%%
%%\mnotex{Double check this is true}
%%\mnotex{EG: I checked in the notebook and only WaveSpinorialCFE4 was used.
%% (the other wave eqs are WaveSpinorialCFE1, WaveSpinorialCFE2 and
%% WaveSpinorialCFE3 which are never used in the derivations )}
%%%%%%%%
This equation reads as follows
\begin{eqnarray}
  \square \phi _{ABCF} = 12 \Lambda \phi _{ABCF} -6 \Xi \phi
  _{(AB}{}^{DG}\phi _{CF)DG}.
  \label{Wave_eq_CFE_Weyl}
\end{eqnarray}
The last equation follows from considering $\nabla^{QC'}\Lambda
_{C'ABC}=0$ and applying the identity
\eqref{DecomposeDoubleDerivativeContracted}.
%%%%%%%%%%%%%%%%%%%%%%%%
%% its derivation following from  $\nabla^{QC'}\Lambda _{C'ABC}=0$
%% by a straightforward calculation using the identity
%% \eqref{DecomposeDoubleDerivativeContracted}.\mnotex{Still needs rewording...}
%%%%%%%%%%%%%%%%%%%%%%%
It is worth noting here that one of the tools used in \cite{GasVal15}
to show the equivalence between the spinorial CFE wave equations and
the original spinorial CFEs
\eqref{Def_ConfFactor_CFE_zeroquant}--\eqref{Def_cons_CFE_zeroquant}
is the uniqueness property of solutions to a certain class of \textit{homogeneous} wave
equations. 

\begin{definition}
{\em An operator $h$ is said to be \textit{homogeneous in} $\underline{u}$
\textit{and its first derivatives} if
$h (\mu\underline{u},\mu\partial\underline{u})=\mu
h(\underline{u}, \partial\underline{u})$ for all
$\mu\in\mathbb{C}$. }
\end{definition}

This same result from the theory of partial differential
equations will be used in the proof of the main theorem of this
article, and so we quote it here:
\begin{theorem}
\label{TheoremHomogeneousWave}
 Let $\mathcal{M}$ be a smooth manifold equipped with a Lorentzian
 metric $\bmg$ and consider the wave equation
\[\square \underline{u}=h (\underline{u},\partial\underline{u})\]
where $\underline{u}\in\mathbb{C}^m$ is a complex vector-valued
function on $\mathcal{M}$, $h:\mathbb{C}^{2m}\rightarrow\mathbb{C}^m$
is a smooth homogeneous function of its arguments and
$\square=g^{ab}\nabla_{a}\nabla_{b}$.  Let
$\mathcal{U}\subset\mathcal{S}$ be an open set and $\mathcal{S}\subset
\mathcal{M}$ be a spacelike hypersurface with normal $\tau^{a}$
respect to $\bmg$. Then the Cauchy problem
\begin{align*}
\square \underline{u}&=h (\underline{u},
\partial\underline{u}),\\ \underline{u}\left|_{\mathcal{U}}\right.&=\underline{u}_0,
\quad
\nabla_{\bm\tau}\underline{u}\left|_{\mathcal{U}}\right.=\underline{u}_1,
\end{align*} 
where $\underline{u}_{0}$ and $\underline{u}_{1}$ are smooth on
$\mathcal{U}$ and $\nabla_{\bm\tau}:= \tau^\mu\nabla_\mu$, has a
unique solution $\underline{u}$ in the domain of dependence of
$\mathcal{U}$.
\end{theorem}
We refer the reader to Proposition 3.2 of \cite{Tay96c} for a proof
---see also Theorem 1 in \cite{GarVal08c}.

\begin{remark}{\em 
Analogous to the physical case, give a Petrov type D $\phi_{ABCD}$, 
one can give an explicit construction of a Killing spinor, namely
\begin{equation*}
%& \kappa_{AB} = o_A o_B && (\textbf{Type N}),\\
\kappa_{AB} = \phi^{-1/3}o_{(A}\iota_{B)} %&& (\textbf{Type D}),
\end{equation*}
in terms of the relevant adapted spin dyad $\lbrace \bmo, \bm\iota\rbrace$. This can be seen directly 
by noting that the identity \eqref{RescaledWeylEquationDisplayed} satisfied by $\phi_{ABCD}$ is formally 
identical to the Bianchi constraint and so the same computations as in \cite{WalkerPenrose70} follow through.
An alternative approach to the construction of Killing spinor initial data equations would be attempt to determine under what conditions the Petrov type of the (rescaled) Weyl tensor restricted to $\mathcal{S}$ is propagated into the spacetime development. Later, we shall see that such a result follows as a product of our analysis ---see Corollary \ref{Corollary:PetrovPropagation}.}\label{Remark:DyadExpressionForKillingSpinorInTypeD}
\end{remark}

  \section{Conformal twistor initial data}
  \label{conformalTwistorKID}
  In this section, the conformal twistor initial data equations are
  derived.  Although the main result of this article is on the
  conformal  Killing spinor initial data equations, the twistor case
    illustrates the main features of the calculation for the Killing
  spinor case of Section \ref{conformalKSKID} in a simpler setting.
\subsection{Twistor zero-quantities}
\label{Sec:TwistorZeroQuantities}

For the following discussion is convenient to make the following
\emph{zero-quantities}
\begin{subequations}
  \begin{eqnarray}
   && H_{A'AB} := 2
    \nabla_{A'(A}\kappa_{B)},\label{Def_H_twistor}\\ && B_{ABC}
    := \phi_{ABCD}\kappa^D.\label{Def_B_twistor}
    \end{eqnarray}
\end{subequations}
The spinors $H_{A'AB}$ and $B_{ABC}$ will be denoted in index free
notation as $\bmH$ and $\bmB$ and will be called the twistor
zero-quantity and the Buchdahl zero-quantity respectively.  The
Buchdahl zero-quantity arises as an integrability condition of the
twistor equation.  To see this, notice that, taking the following
derivative of $\bmH$ and substituting definition
\eqref{Def_H_twistor} one obtains
  \begin{equation}\label{curl_H_twistor}
  \nabla_{AA'}H^{A'}{}_{BC}= 2 \nabla_{AA'}\nabla_{(B}{}^{A'}\kappa
  _{C)} = \tfrac{1}{2} \epsilon _{AB} \square \kappa _{C}  +
  \tfrac{1}{2}  \epsilon _{AC} \square \kappa _{B} +
  \square_{BA}\kappa _{C} + \square_{CA}\kappa _{B}.
  \end{equation}
  Symmetrising and using equation \eqref{SpinorialRicciIdentities1} gives
  \[
  \nabla_{(A|A'|}H^{A'}{}_{BC)}= - 2\Psi_{ABCD}\kappa^D.
  \]
  The vanishing of the right-hand side of latter equation encodes the
  Buchdahl constraint, namely the fact that if $(\mathcal{M},\bmg)$
  admits a twistor then it is necesarilly of Petrov type N or O. To
  write this in the variables appearing in the CFEs,
  using the definition of the rescaled Weyl spinor
  yields
  \begin{equation}\label{Curl_H_sym_toB_twistor}
  \nabla_{(A}{}^{A'}H_{|A'|BC)} = 2\Xi B_{ABC},
  \end{equation}
  which motivates the name for the zero-quantity $\bmB$.
  Thus, with this notation, it is clear that if the unphysical spacetime
  $(\mathcal{M},\bmg)$ admits a twistor  then following
  zero-quantities vanish
  \begin{equation}
H_{A'AB}=0, \qquad B_{ABC}=0.
  \end{equation}
  
\subsection{Twistor auxiliary quantities and the twistor candidate equation}
  
 A useful bookkeeping device for the subsequent
calculations are the definitions of the following \emph{auxiliary quantities}:
\begin{subequations}\label{def_twistor_aux_quants}
  \begin{eqnarray}
      && Q_{A}  := \nabla^{QA'}H_{A'QA}, \label{def_Q_twistor} \\
      && \xi_{A'} := \nabla^B{}_{A'}\kappa_B. \label{def_xi_twistor}
  \end{eqnarray}
\end{subequations}
  The \emph{auxiliary spinor} $\xi_{A'}$ is merely a convenient
  placeholder for making irreducible decompositions of derivatives of
  $\kappa_A$, such as
  \begin{align}\label{decomp_Der_kappa}
    \nabla_{AA'}\kappa _{B} & = \tfrac{1}{2} \epsilon _{AB}
    \nabla_{CA'}\kappa ^{C} + \nabla_{(A|A'|}\kappa _{B)} = \tfrac{1}{2} H_{A'AB} - \tfrac{1}{2} \xi _{A'} \epsilon_{AB}.
  \end{align}
It is nevertheless illustrative to introduce this
  shorthand since the analogous quantity in the Killing spinor case
  will play an important role in the calculation.

  \medskip
  
  On the other hand, the \emph{auxiliary quantity}
  $Q_A$ will be central for the following discussion since it
  encodes a wave equation for $\kappa_A$. To see this, observe that
  tracing the identity \eqref{curl_H_twistor} and substituting 
  definition \eqref{def_Q_twistor} gives
\begin{equation}\label{Q_to_box_twistor_candidate}
Q_{A} = 3 \Lambda \kappa _{A} + \tfrac{3}{2} \square \kappa _{A}.
\end{equation}
Solving for $\square \kappa _{A}$ one has
\[
\square \kappa _{A} = \tfrac{2}{3} Q_{A} -2 \Lambda \kappa _{A}.
\]
If the equation $Q_{A}=0$ is imposed, then
the latter expression can be read as a wave equation for $\kappa_A$.
This motivates the following definition: a valence-1 spinor $\eta_A$ satisfying
\begin{align} \label{Wave_eq_twistor_candidate}
\square \eta _{A} = -2 \Lambda  \eta _{A}
\end{align}
will be called a \emph{twistor candidate}. To understand the
motivation for this definition and its name, notice that in general,
any twistor $\kappa_A$ trivially satisfies the twistor candidate
equation but not every twistor candidate $\eta_A$ will solve the
twistor equation. In other words,
\[
\bmH=0 \implies \bmQ =0, \qquad \text{but in general} \qquad \bmQ =0 \notimplies \bmH=0.
\]
However,  the initial data $(\eta_A, \nabla_{\bm\tau}
\eta_A)|_{\mathcal{S}}$ for the wave equation
\eqref{Wave_eq_twistor_candidate} has not been fixed yet. The aim of the following
calculations is to determine the conditions on the initial data for the
twistor candidate such that if propagated off $\mathcal{S}$, using
equation \eqref{Wave_eq_twistor_candidate}, then the corresponding twistor
candidate $\eta_A$ is, in fact, a twistor. Namely,
\begin{equation}
\bmQ =0 \;\;\&\;\; \text{twistor initial data} \;\;\implies \bmH=0.
\end{equation}
The strategy to obtain such conditions on the intial data
$(\nabla_{\bm\tau} \eta_A, \eta_A)|_{\mathcal{S}}$  is to derive a closed
system of homogeneous wave equations for the zero-quantities $\bmH$
and $\bmB$ to show that, if trivial initial data for such
equations is given, then, using Theorem \ref{TheoremHomogeneousWave},
$\bmH=0$ and $\bmB=0$ in the domain of dependence of the data.

\subsection{Wave equations for the zero-quantities}

A wave equation for the zero-quantity $\bmH$ can be constructed as
follows.  From the irreducible decomposition of
$\nabla_D{}^{A'}H_{A'AB}$,
\[
\nabla_{D}{}^{A'}H_{A'AB} = \tfrac{1}{3} \epsilon _{BD}
\nabla_{CA'}H^{A'}{}_{A}{}^{C} + \tfrac{1}{3} \epsilon _{AD}
\nabla_{CA'}H^{A'}{}_{B}{}^{C} + \nabla_{(A}{}^{A'}H_{|A'|BD)},
\]
 along with definition \eqref{def_Q_twistor} and equation
\eqref{Curl_H_sym_toB_twistor} it follows that
\begin{align}\label{derH_twistor_toBandQ}
\nabla_{D}{}^{A'}H_{A'AB} = 2 \Xi B_{ABD}  + \tfrac{1}{3} Q_{B}
\epsilon _{AD} + \tfrac{1}{3} Q_{A} \epsilon _{BD}.
\end{align}
Applying $\nabla_{D}{}^{B'}$ to the last expression, and using
identity \eqref{DecomposeDoubleDerivativeContracted} along with the
spinorial Ricci identities
\eqref{SpinorialRicciIdentities1}-\eqref{SpinorialRicciIdentities2},
renders
\begin{equation}\label{wave_H_twistor}
  \square H_{B'AB} = 6 \Lambda H_{B'AB} + 4 \Xi
  \nabla_{DB'}B_{AB}{}^{D} -4 B_{ABD} \nabla^{D}{}_{B'}\Xi -4
  \Phi_{(A}{}^{D}{}_{|B'}{}^{A'}H_{A'|B)D} + \tfrac{4}{3}
  \nabla_{(A|B'|}Q_{B)}.
\end{equation}
To derive a wave equation for $\bmB$, one applies the D'Alembertian operator
$\square$ to the definition \eqref{Def_B_twistor} to obtain
\begin{align}\label{pre_wave_B_twistor}
\square B_{ABC} = \kappa ^{D} \square \phi _{ABCD} + \phi _{ABCD}
\square \kappa ^{D} + 2 \nabla_{FA'}\phi _{ABCD} \nabla^{FA'}\kappa
^{D}.
\end{align}
Substituting the definition \eqref{Def_B_twistor}, the identity
\eqref{Q_to_box_twistor_candidate}, and the wave equation satisfied by
the rescaled Weyl spinor \eqref{Wave_eq_CFE_Weyl} into the last expression gives
\begin{equation}\label{wave_B_twistor}
\square B_{ABC} = 10 \Lambda B_{ABC}  + H^{A'DF} \nabla_{FA'}\phi _{ABCD}  -6 \Xi B_{(A}{}^{DF}\phi
_{BC)DF} + \tfrac{2}{3} \phi _{ABCD} Q^{D}.
\end{equation}
Observe that if $Q_{A}=0$, namely if the twistor candidate wave equation is imposed, then,
$\bmH$ and $\bmB$ satisfy the following set of wave equations
\begin{subequations}
\begin{eqnarray}
  && \square H_{B'AB} = 6 \Lambda H_{B'AB} + 4 \Xi
  \nabla_{DB'}B_{AB}{}^{D}  -4 B_{ABD} \nabla^{D}{}_{B'}\Xi   -4 \Phi_{(A}{}^{D}{}_{|B'}{}^{A'}H_{A'|B)D},
   \label{Hom_wave_HandB1} \\
 && \square B_{ABC} = 10\Lambda B_{ABC} + H^{A'DF} \nabla_{FA'}\phi _{ABCD}  -6 \Xi B_{(A}{}^{DF}\phi
_{BC)DF}.  \label{Hom_wave_HandB2}
\end{eqnarray}
\end{subequations}
Notice that the only place where the CFEs ---in their wave equation form---
have been used was in equation \eqref{pre_wave_B_twistor}
to substitute for the term $\square \phi _{ABCD}$.

The relevant observation about equations
\eqref{Hom_wave_HandB1}-\eqref{Hom_wave_HandB2} is that they
comprise a closed system of \emph{regular and homogeneous}
wave equations for $\bmH$ and $\bmB$.
Hence prescribing trivial intial data
\[
H_{A'AB}=0, \qquad \nabla_{\bm\tau} H_{A'AB}=0, \qquad B_{ABC}=0, \qquad \nabla_{\bm\tau} B_{ABC}=0 \qquad \text{on} \qquad \mathcal{S}
\]
and using Theorem \ref{TheoremHomogeneousWave}, which establishes the
uniquess of solutions to wave equations of the type
of \eqref{Hom_wave_HandB1}-\eqref{Hom_wave_HandB2}, one has that 
\begin{equation}\label{ID_trivial_H_B_twistor}
H_{A'AB}=0, \qquad B_{ABC}=0 \qquad \text{on} \qquad \mathcal{D}^{+}(\mathcal{S}) .
\end{equation}
In turn, substituting the definitions for the zero-quantities $\bmH$
and $\bmB$ into the conditions \eqref{ID_trivial_H_B_twistor} render a
prescription to fix the intial data $(\nabla_{\bm\tau} \eta_A,
\eta_A)|_\mathcal{S}$ for  twistor
candidate wave equation \eqref{Wave_eq_twistor_candidate} that ensures that
 the twistor candidate $\eta_A$ will
correspond to an actual twistor $\kappa_A$ in $\mathcal{D}^{+}(\mathcal{S})$.
\\

This discussion is summarised in the following:
\begin{proposition}\label{Prop:Propagation_twistor}
  Given initial data for the conformal field equations on $\mathcal{U}\subseteq\mathcal{S}$
  where $\mathcal{S}$ is a spacelike
hypersurface $\mathcal{S}$ with normal vector $\tau^{AA'}$, and
associated normal derivative $\nabla_{\bm\tau} :=
\tau^{AA'}\nabla_{AA'}$, the corresponding spacetime development
admits a twistor (valence-1 Killing spinor) in $\mathcal{D}^{+}(\mathcal{U})$
if and only if
\begin{subequations}
\begin{eqnarray}
  && H_{A'AB}=B_{ABC}=0,\label{eq:VanishingOfH_twistor}\\ 
  %&& \nabla_{\bm\tau}H_{A'AB}=0,\label{eq:VanishingOfNormalDerivH_twistor}\\ 
  %&&B_{ABC}=0,\label{eq:VanishingOfB_twistor}\\ 
  &&\nabla_{\bm\tau}H_{A'AB}=\nabla_{\bm\tau}
  B_{ABC}=0, \label{eq:VanishingOfNormalDerivB_twistor}
\end{eqnarray}
\end{subequations}
 hold on $\mathcal{U}$.
\end{proposition}
\begin{proof}
The \emph{only if} direction is immediate. Suppose, on the other hand,
that
\eqref{eq:VanishingOfH_twistor}-\eqref{eq:VanishingOfNormalDerivB_twistor}
hold on some $\mathcal{U}\subset\mathcal{S}$ ---that is to say, there
exist a spinor field $\kappa_{A}$ for which
\eqref{eq:VanishingOfH_twistor}-\eqref{eq:VanishingOfNormalDerivB_twistor}
are satisfied on $\mathcal{U}$. The latter is then used as initial
data for the twistor candidate wave equation
\begin{align} \label{Wave_eq_twistor_candidate_prop}
\square \kappa _{A} = -2 \Lambda \kappa _{A}.
\end{align}
As the zero-quantities $H_{A'AB},~B_{ABC}$ satisfy the homogeneous
wave equations \eqref{Hom_wave_HandB1}-\eqref{Hom_wave_HandB2} then
the uniqueness result for homogeneous wave equations, given in
Theorem \ref{TheoremHomogeneousWave},
ensures that
\[ H_{A'ABC}=0,\qquad B_{ABC}=0,\]
in $\mathcal{D}^{+}(\mathcal{U})$. In other words,
$\kappa_{A}$ solves the twistor equation on $\mathcal{D}^{+}(\mathcal{U})$.
\end{proof}

\begin{remark}
\em{
For the twistor case, one important difference between discussion in
\cite{GasVal15}
%%---performed in the physical framework---
using the vacuum Einstein field equations in  $(\tilde{\mathcal{M}},\tilde{\bmg})$
is that the system closes with $\tilde{H}_{A'AB}$ alone and there is no need
to introduce the analogous physical Buchdahl zero-quantity $\tilde{B}_{ABC}$.
Therefore it is interesting to check if in the conformal case
%%---unphysical framework---
one can also close the system with $H_{A'AB}$ alone.
To do so, observe that, substituting the expression for
the Buchdahl constraint into equation
\eqref{Hom_wave_HandB1} and using the CFEs, gives
\begin{multline}\label{WaveH_twistor_singular}
  \square H_{A'AB} =  - 2\Xi^{-1} (\nabla^{C}{}_{A'}\Xi)
    \nabla_{(A}{}^{B'}H_{|B'|BC)}+ 6 \Lambda H_{A'AB} \\-2 \Xi
  \phi _{ABCD}H_{A'}{}^{CD} -4
  \Phi_{(A}{}^{C}{}_{|A'}{}^{B'}H_{B'|B)C}.
  %%%%%
  %\nonumber \\ & +
  %\tfrac{4}{3} \nabla_{(A|A'|}Q_{B)}.
  %%%%
\end{multline}
%%%%%%%%
%\noindent
% Hence, setting $Q_{A}=0$,
%%%%%%%
Hence, $\bmH$ satisfies  an homogeneous but
\emph{singular equation}, due to the $\Xi^{-1}$ coefficient,
for $H_{A'AB}$, a situation to which Theorem
\ref{TheoremHomogeneousWave} does not apply.
From equation \eqref{WaveH_twistor_singular} one can
recover the analogous wave equation in the physical case
discussed in \cite{GasVal15} simply by adding a tilde to the fields and
setting $\Xi=1$.

Arguably, one could try
to use the theory of \textit{Fuchsian systems}, as used in \cite{ChrPaetz13,Pae14a}, to see if the analogue of
Theorem \ref{TheoremHomogeneousWave} applies for the singular equation
\eqref{WaveH_twistor_singular}.  However, one of the advantages of the
conformal approach of the CFEs is that one deals with manifestly regular equations.
Therefore, from this perspective, it is necessary to introduce $B_{ABC}$ as a further zero-quantity to
be propagated.  A analogous observation holds for the conformal valence-2
Killing spinor initial data discussion of the following sections,
where, to close the system in a regular way, one needs to introduce not
only the Buchdahl zero-quantity but also its derivative.}
\end{remark}


\subsection{Intrinsic conformal twistor initial data conditions}

In this section, the conformal twistor initial data conditions of
proposition \ref{Prop:Propagation_twistor} are written in terms of
intrinsic quantities at $\mathcal{S}$.  To understand the need for the
calculation to be carried out in this section observe that, although
the conditions of Proposition \ref{Prop:Propagation_twistor} are given
on $\mathcal{S}$, they contain not only derivatives tangential to
$\mathcal{S}$ but also normal to it. Hence, to obtain genuine
intrinsic conditions on $\mathcal{S}$ one needs to remove these normal
derivatives.

\medskip

Introducing the following definitions:
\begin{align}
  %H_{CAB} \eqref \tau _{C}{}^{A'} H_{A'AB}, \qquad
  %\mathcal{H} _{ABC} := H_{(ABC)},
  %\qquad \mathcal{H} _{A} := H^{D}{}_{AD}.
  \mathcal{H} _{ABC}  := \tau _{(A}{}^{A'}H_{|A'|BC)}, \qquad
  \mathcal{H}_{A}  :=  \tau^{QA'} H_{A'AQ},
\end{align}
the space spinor split of $H_{A'AB}$ reads
\begin{align}
  H_{A'AB} = - \tfrac{1}{2} \tau ^{C}{}_{A'} \mathcal{H} _{ABC}  -
  \tfrac{1}{6} \tau ^{C}{}_{A'} \mathcal{H} _{B} \epsilon _{AC}  -
  \tfrac{1}{6} \tau ^{C}{}_{A'} \mathcal{H} _{A} \epsilon _{BC}.
\end{align}
Hence, the space spinors $\mathcal{H} _{ABC}$ and $\mathcal{H}_{A}$
contain all the information of $H_{A'AB}$. In other words,
\[
H_{A'ABC}=0 \quad                 
\iff \quad \mathcal{H} _{A}=0    
\quad
\& \quad \mathcal{H}_{ABC}=0  
\]
Substituting the definition \eqref{Def_H_twistor} one obtains
\begin{align}\label{spacespinordecompHtotwistorders}
\mathcal{H} _{A} = \tfrac{3}{2} \nabla_{\bm\tau} \kappa_{A} - \mathcal{D} _{AB}\kappa^{B}, \qquad \mathcal{H} _{ABC} = 2 \mathcal{D} _{(AB}\kappa _{C)},
\end{align}
Then, $H_{A'AB}|_{\mathcal{S}}=0$  imposes the
following conditions on the
the initial data
$(\kappa_A,\nabla_{\bm\tau}\kappa_A)|_{\mathcal{S}}$
for the twistor candidate wave equation
\eqref{Wave_eq_twistor_candidate_prop}:
\begin{align}\label{H_twistor_vanishes_ID}
 \nabla_{\bm\tau} \kappa _{A} = \tfrac{2}{3} \mathcal{D} _{AB}\kappa ^{B}, \qquad
 \quad \mathcal{D} _{(AB}\kappa _{C)}=0 \qquad \text{on} \qquad \mathcal{S}.
\end{align}
Another set of constraints arise from the conditions $\nabla_{\bm\tau}
H_{A'BC}|_{\mathcal{S}}=0$ and $B_{ABC}|_{\mathcal{S}}=0$. These two
conditions can be analysed in tandem since $\bmB$ is related to the
derivative of $\bmH$. Using the space spinor split of $\nabla$ it
follows, exploting the identity \eqref{derH_twistor_toBandQ}, that
\begin{align}
  \tau _{D}{}^{A'}\nabla_{\bm\tau} H_{A'AB} -2 \tau ^{CA'} \mathcal{D}
  _{DC}H_{A'AB} = 4 B_{ABD} \Xi + \tfrac{4}{3} Q_{(A}\epsilon
  _{B)D}\quad
\end{align}
Transvecting with $\tau^{D}{}_{B'}$ and rearranging gives
\begin{align}
\nabla_{\bm\tau} H_{B'AB} = -4 B_{ABD} \Xi \tau ^{D}{}_{B'} -2 \tau ^{CA'}
\tau ^{D}{}_{B'} \mathcal{D} _{DC}H_{A'AB} - \tfrac{4}{3} \tau
^{D}{}_{B'}Q_{(A}\epsilon _{B)D}
\end{align}
Hence, if the the twistor candidate wave equation is imposed, namely
$Q_A=0$, then
\begin{align}
H_{A'AB}|_{\mathcal{S}}=0\quad \& \quad B_{ABC}|_{\mathcal{S}}=0
\implies \nabla_{\bm\tau} H_{A'AB}|_{\mathcal{S}}=0.
\end{align}
In other words, imposing $\nabla_{\bm\tau} H_{A'AB}|_{\mathcal{S}}=0$ is
redundant if $H_{A'AB}|_{\mathcal{S}}=0$ and $
B_{ABC}|_{\mathcal{S}}=0$ are satisfied.  Using the definition
\eqref{Def_B_twistor}, the condition $B_{ABC}|_{\mathcal{S}}=0$ simply
reads,
\begin{align}
  \phi_{ABCD}\kappa^D=0 \qquad \text{on} \qquad \mathcal{S}.
\end{align}
Finally, for the condition $\nabla_{\tau}B_{ABC}|_{\mathcal{S}}=0$ one
has, applying $\nabla_{\bm\tau}$ to equation \eqref{Def_B_twistor} that
\begin{align}
\nabla_{\bm\tau} B_{ABC} = \phi _{ABCD}\nabla_{\bm\tau} \kappa ^{D} + \kappa
^{D} \nabla_{\bm\tau} \phi _{ABCD} .
\end{align}
At this point one can exploit the evolution equation for the rescaled
Weyl spinor \eqref{RescaledWeyl_evo_const} to subsitute for
$\nabla_{\bm\tau} \phi_{ABCD}$ and using equation
\eqref {spacespinordecompHtotwistorders}
to substitute $\nabla_{\bm\tau} \kappa_A$ when evaluating at
$\mathcal{S}$. This substitution renders,
%the condition
%$\nabla_{\tau}B_{ABC}|_{\mathcal{S}}=0$
%reads
\begin{align}\label{normalderB_twistor_exp}
\nabla_{\tau}B_{ABC}|_{\mathcal{S}}= -2\kappa ^{D} \mathcal{D}
_{DF}\phi _{ABC}{}^{F} + \tfrac{2}{3} \phi _{ABCD} \mathcal{D}
^{D}{}_{F}\kappa ^{F} + \tfrac{2}{3}\phi_{ABCD}\mathcal{H}^D = 0
\qquad \text{on} \qquad \mathcal{S}.
\end{align}
In fact, the latter expression can be completely rewritten in terms of
$\mathcal{H}_A|_{\mathcal{S}}$, $\mathcal{H}_{ABC}|_{\mathcal{S}}$ and
$B_{ABC}|_{\mathcal{S}}$. This can be done as follows: swapping
indices $D$ and $A$ in equation \eqref{normalderB_twistor_exp}, and
exploiting the constraint equation for the rescaled Weyl spinor in
expression \eqref{RescaledWeyl_evo_const} gives
\begin{align}\label{normalderB_twistor_exp2}
\nabla_{\tau}B_{ABC}|_{\mathcal{S}}= -2 \kappa ^{D} \mathcal{D}
_{AF}\phi _{DBC}{}^{F} + \tfrac{2}{3} \phi _{ABCD} \mathcal{D}
^{D}{}_{F}\kappa ^{F} + \tfrac{2}{3}\phi_{ABCD}\mathcal{H}^D = 0
\qquad \text{on} \qquad \mathcal{S}.
\end{align}
Applying a $\mathcal{D}_{FQ}$ to the definition \eqref{Def_B_twistor}
and using the Leibnitz rule, one can replace the first term in the
last equation to obtain
\begin{multline}\label{normalderB_twistor_exp3}
\nabla_{\tau}B_{ABC}|_{\mathcal{S}}= -2 \mathcal{D} _{AD}B_{BC}{}^{D}
-2 \phi _{BCDF} \mathcal{D} _{A}{}^{F}\kappa ^{D} \\+\tfrac{2}{3} \phi
_{ABCF} \mathcal{D} _{D}{}^{F}\kappa ^{D} +
\tfrac{2}{3}\phi_{ABCD}\mathcal{H}^D = 0 \quad \text{on} \quad
\mathcal{S}.
\end{multline}
From the irreducible decomposition of $\mathcal{D} _{AB}\kappa _{C}$
and using the expression for $\mathcal{H}_{ABC}$ equation
\eqref{spacespinordecompHtotwistorders} one has
\begin{align}\label{decompSenKappa}
\mathcal{D} _{AB}\kappa _{C} = \tfrac{1}{2} \mathcal{H} _{ABC} +
\tfrac{1}{3} \epsilon _{BC} \mathcal{D} _{AD}\kappa ^{D} +
\tfrac{1}{3} \epsilon _{AC} \mathcal{D} _{BD}\kappa ^{D}.
\end{align}
Substituting decomposition \eqref{decompSenKappa} into equation
\eqref{normalderB_twistor_exp3} gives
\begin{align}
\nabla_{\tau}B_{ABC}|_{\mathcal{S}}=- \phi _{BCDF} \mathcal{H}
_{A}{}^{DF} -2 \mathcal{D} _{AD}B_{BC}{}^{D} +
\tfrac{2}{3}\phi_{ABCD}\mathcal{H}^D = 0 \quad \text{on} \quad
\mathcal{S}
\end{align}
Hence, overall, the only independent conditions to be imposed are
$H_{A'AB}|_{\mathcal{S}}=0$ and $B_{ABC}|_{\mathcal{S}}=0$.  \\

The discussion of this section can be summarised in the following:
\begin{theorem}\label{Theorem_twistor}
Consider an initial data set for the vacuum conformal Einstein
field equations, as encoded in the CFE zero-quantities
\eqref{Def_ConfFactor_CFE_zeroquant}--\eqref{Def_cons_CFE_zeroquant},
on a spacelike hypersurface $\mathcal{S}$ and let
$\mathcal{U}\subset\mathcal{S}$ be an open set.
The development
of the initial data set will have a twistor (valence-1 Killing spinor)
in the domain of dependence of $\mathcal{U}$ if and only if there
exists a solution $\kappa_A$ to the equations
\begin{flalign}
  \mathcal{D} _{(AB}\kappa _{C)}=0 , %\label{CS-KID1_twistor}
    \qquad
   \phi_{ABCD}\kappa^D=0, \label{CS-KID_twistor}
\end{flalign}
on $\mathcal{U}$. The twistor is obtained by evolving according
to the wave equation:
\begin{align} \label{Wave_eq_twistor_candidate_theo}
\square \kappa _{A} = -2 \Lambda \kappa _{A}.
\end{align}
with initial data satisfying conditions in equation
\eqref{CS-KID_twistor} and
\begin{equation}
  \nabla_{\bm\tau} \kappa _{A} = \tfrac{2}{3} \mathcal{D} _{AB} \kappa^B.
\end{equation}
\end{theorem}

\begin{proof}
The analysis of the last subsection shows that conditions
\begin{eqnarray*}
 H_{A'AB}=0, \quad \nabla_{\bm\tau} H_{A'AB}=0, \quad B_{ABC}=0, \quad
 \nabla_{\bm\tau} B_{ABC}=0
\end{eqnarray*}
on $\mathcal{U}\subset \mathcal{S}$ are equivalent to the conditions
\eqref{CS-KID_twistor}.  Hence, using Proposition
\ref{Prop:Propagation_twistor} one concludes that if equations
\eqref{CS-KID_twistor} hold on $\mathcal{U}$, then the domain of
dependence of $\mathcal{U}$ is endowed with a twistor.
\end{proof}


  
\section{Conformal Killing spinor initial data}
  \label{conformalKSKID}

\mnotex{change vert S to vert U everywhere}


In this section, the conformal (valence-2) Killing spinor initial data
equations are derived.  Although the calculations will be more
involved, in general terms, we follow the same strategy discussed in
section \ref{conformalTwistorKID} for the twistor case.
%%%%%%%%%%%%%%%%%%%
%% Broadly speaking, we will follow the same
%% strategy as in the twistor case, although the calculations will be
%% somewhat more involved.
%%%%%%%%%%%%%%%%

\subsection{Killing spinor zero quantities}

Analogous to the twistor case, we define the zero quantities:
\begin{equation}\label{KS_zero_quantities1}
H_{A'ABC}:=3\nabla_{A'(A}\kappa_{BC)}, \qquad B_{ABCD}:=\kappa_{(A}{}^Q\phi_{BCD)Q}.
\end{equation}
A short computation shows that
\begin{equation}
\nabla_{(A}{}^{A'}H_{\vert A'\vert BCD)} = 6\Xi
B_{ABCD}. \label{Eq:BuchdahlAsCurlOfH}
\end{equation}
%%%
%We will see that,
%%%
Despite the formal resemblence with the equations of section
\ref{Sec:TwistorZeroQuantities}, the following discussion
will show that, unlike the twistor case, one cannot obtain a closed
homogeneous wave system in terms of these variables alone and we shall
need the following additional zero quantity
\begin{equation}
 F_{A'BCD}:=\nabla^Q{}_{A'}B_{QBCD}.\label{Eq:DefZeroQuantityF}
\end{equation}
In the sequel, it wil be shown that using the enlarged collection of variables
 defined by expressions \eqref{KS_zero_quantities1} and \eqref{Eq:DefZeroQuantityF},
it is possible to derive
%%%%%%%%%%%%
%w% e shall see that one can derive
%%%%%
a closed homogeneous wave system for $(\bmH, \bmB, \bmF)$. 


\subsection{Auxiliary quantities and the Killing spinor candidate equation}

By analogy with the twistor case, it will prove useful to define the
following auxiliary quantity,
\begin{equation}
  \mathcal{Q}_{BC} := \tfrac{1}{2}\nabla^{AA'}H_{A'ABC}.
%%:= 2(\square \kappa_{BC} + 4
%%\Lambda\kappa_{BC} - \Xi
%%\phi_{BCAD}\kappa^{AD})=0.
  \label{Eq:WaveForKS_DefinitionQ}
\end{equation}
A direct calculation shows that the auxiliary quantity $\mathcal{Q}_{AB}$ encodes
a wave equation for $\bm\kappa$.
\begin{equation}
  \mathcal{Q}_{BC} = \square \kappa_{BC} + 4 \Lambda\kappa_{BC} - \Xi
  \phi_{BCAD}\kappa^{AD}. %=0.
  \label{Eq:WaveForKS}
\end{equation}
The wave equation for the Killing spinor candidate will then
be imposed by setting $\mathcal{Q}_{AB}=0$. It will also prove convenient to
define the auxiliary spinor $\xi_{AA'} :=
\nabla^{B}{}_{A'}\kappa_{AB}$ in terms of which one can perform the
following decomposition 
\begin{equation}
\nabla_{AA'}\kappa_{BC} = \tfrac{1}{3} H_{A'ABC} - \tfrac{1}{3}
\xi_{CA'} \epsilon_{AB} - \tfrac{1}{3} \xi_{BA'}
\epsilon_{AC}.\label{Eq:DecompGradKS}
\end{equation}
Using the latter expression, one can show by a straightforward
computation that
\begin{equation}\label{eq:non-KillingVectorExplained}
  \nabla^Q{}_{A'}H_{B'ABQ} +  \mathcal{Q}_{AB} \epsilon_{A'B'} = \nabla_{AA'}\xi_{BB'} + \nabla_{BB'}\xi_{AA'} + 6 \kappa_{(A}{}^{Q} \Phi_{B)QA'B'},
\end{equation}
which in turn implies the following identity
\begin{multline}
\nabla_{AA'}\xi_{BB'} = - \tfrac{1}{2} \epsilon_{AB}
\nabla^{C}{}_{(A'}\xi_{|C|B')}- 3 \kappa_{(A}{}^{C} \Phi_{B)CA'B'} - 3
\Lambda \kappa_{AB} \epsilon_{A'B'} \\+ \tfrac{3}{4} \Xi \kappa^{CD}
\phi_{ABCD} \epsilon_{A'B'} + \tfrac{1}{8} Q_{AB} \epsilon_{A'B'} +
\tfrac{1}{2} \nabla^Q{}_{(A'}H_{B')ABC},\label{Eq:DecompGradXi}
\end{multline}
which will prove useful later. On the other hand, it is
straightforward to show using equation \eqref{Eq:BuchdahlAsCurlOfH} that
\begin{equation}
     \nabla_{D}{}^{A'}H_{A'ABC} = 6 \Xi B_{ABCD} + \tfrac{3}{2} \
\mathcal{Q}_{(AB}\epsilon_{C)D}.\label{IrrDecompCurlOfH}
\end{equation}



\begin{remark}
  \emph{ As stressed in Section \ref{sec:Introduction}, in general,
  the auxiliary spinor $\xi_{AA'}$ is not the spinorial counterpart of
  a Killing vector.  Contrast this with the case of the physical
  framework ---namely $(\tilde{\mathcal{M}},\tilde{\bmg})$ satisfying
  the vacuum Einstein field equations--- where the last term in
  equation \eqref{eq:non-KillingVectorExplained} vanishes and hence
  $\tilde{\bm\xi}$ becomes a Killing vector. This point is subtle even
  in the physical framework if one considers matter models such the
  Einstein-Maxwell since at this point it is necessary to make further assumptions
  such as the matter alignment condition to ensure that $\tilde{\bm\xi}$ is a
  Killing vector ---see \cite{ValCol16} for details. This property of
  $\tilde{\bm\xi}$ is crucial for the derivation of the physical
  Killing spinor data equations presented in \cite{GarVal08c} and
  \cite{ValCol16} since the associated zero-quantity $\tilde{S}_{ab}:=
  \tilde{\nabla}_{(a}\tilde{\xi}_{b)}$ is introduced. Moreover, in the
  physical framework one imposes evolution equations not only for the
  Killing spinor candidate $\tilde{\bm\kappa}$ but also for the
  Killing vector candidate $\tilde{\bm\xi}$.  In the unphysical
  framework one cannot appeal to this strategy since the unphysical
  Ricci spinor $\bm\Phi$ is non-vanishing and does not satisfy any
  algebraic relation as in the physical case.
%%%%%%%%%%%%%%%%%%%%%%%%%%%%%%%
  %such as those imposed by the matter alignment
  %condition in the physical Einstein-Maxwell case.
%%%%%%%%%%%%%%%%%%%%%%%%%%%%%%%
  As we will see in
  the following, the key to solve this problem in the unphysical
  framework is not to introduce the analogous quantity $S_{ab}$ nor
  imposing a candidate wave equation for $\bm\xi$ but, instead, to
  focus on the Buchdahl constraint $\bmB$ and its derivative $\bmF$.
  }
\end{remark}


\begin{remark} \emph{
  If $\tilde{\kappa}_{AB}$ is a Killing spinor in the the physical
  spacetime $(\tilde{\mathcal{M}},\tilde{\bmg})$ then
  $\tilde{\xi}_{AA'} := \nabla_A{}^{Q'}\tilde{\kappa}_{BA'}$ is a
  Killing vector in the physical spacetime.  It is direct to check
  that $X_{AA'}=\Xi^2 \tilde{\xi}_{AA'}$ is a conformal Killing vector
  in the unphysical spacetime $(\mathcal{M},\bmg)$ with
  $\bmg=\Xi^2\tilde{\bmg}$.  Hence, given that
  $\kappa_{AB}=\Xi^{2}\tilde{\kappa}_{AB}$ a calculation using the
  conformal transformation rules for the connection gives
\begin{equation}\label{eq:conformalKillingVector}
X_{AA'}=\Xi \xi_{AA'} - 3 \kappa_{AQ}\nabla_{A'}{}^{Q}\Xi
\end{equation}
Alternatively, one can verify that expression \eqref{eq:conformalKillingVector}
corresponds to a conformal Killing vector of $(\mathcal{M},\bmg)$ as follows.
Using equations \eqref{eq:conformalKillingVector} and
\eqref{Eq:DecompGradKS} and the equation satisfied by the conformal
factor $\Xi$ encoded in the CFE zero quantity
\eqref{Def_ConfFactor_CFE_zeroquant}, a calculation gives
%%%%%%%%%%%%%
%% \begin{multline}\label{conformalKillingvector}
%%   \nabla_{AA'}X_{BB'}+\nabla_{BB'}X_{AA'}-\tfrac{1}{2}\epsilon_{AB}\epsilon_{A'B'}\nabla^{CC'}X_{CC'}
%%   =-\Xi (\nabla^Q{}_{A'}H_{B'ABQ} + \tfrac{1}{2} Q_{AB}
%%   \epsilon_{A'B'}) \\ + 2H_{(A'|(AB)}{}^{C}\nabla_{C|B')}\Xi.
%% \end{multline}
%%%%%%%%%%%%%%%%%%%%
\begin{multline}\label{conformalKillingvector}
  \nabla_{AA'}X_{BB'}+\nabla_{BB'}X_{AA'}-\tfrac{1}{2}\epsilon_{AB}\epsilon_{A'B'}\nabla^{CC'}X_{CC'}
  =-\Xi \nabla^Q{}_{(A'}H_{B')(AB)Q} + 2\nabla_{C(A'}H_{B')AB}{}^{C}.
\end{multline}
Therefore, if $\bm\kappa$ is a Killing spinor of $(\mathcal{M},\bmg)$ then
the zero-quantities $\bmH$ and $\bmQ$ vanish and hence the right-hand
side of equation \eqref{conformalKillingvector} vanishes, implying
that $\bmX$ is a conformal Killing vector for the unphysical spacetime $(\mathcal{M},\bmg)$.  }
\end{remark}

\subsection{Wave equations for the zero-quantities}
\label{Sec:KSWaveEqs}

Applying $\nabla^A{}_{B'}$ to equation
\eqref{eq:non-KillingVectorExplained} one obtains the following wave
equation:
\begin{multline}
    \square H_{B'ABC} = 6 \Lambda H_{B'ABC} - 12 \Xi F_{B'ABC} - 12\nabla^{D}{}_{B'}\Xi  B_{ABCD} \\+ 3\nabla_{(A|B'|}\mathcal{Q}_{BC)} - 6 \Phi_{(A}{}^{D}{}_{|B'}{}^{A'}H_{A'|BC)D}.
\end{multline}
Similarly, applying $\nabla_A{}^{A'}$ to equation
\eqref{Eq:DefZeroQuantityF}, it is straightforward to verify the
following wave equation for $B_{ABCD}$: \mnotex{I haven't carried the
  $\bmQ$ quantities}
\begin{equation}
     \square B_{ABCD} = 12\Lambda B_{ABCD} - 6\Xi
     \phi_{(AB}{}^{FG}B_{CD)FG} +
     2\nabla_{AA'}F^{A'}{}_{BCD}. \label{Eq:FirstWaveEqForB}
\end{equation}
The task remaining is to derive a wave equation for $F_{A'ABC}$.  To
do so, we will need some ancilliary identities first.
A direct calculation shows that
\begin{multline}
2\phi_{(AB}{}^{GH}B_{CF)GH}=
\kappa_A{}^D\phi_{(BC}{}^{GH}\phi_{FD)GH}+\kappa_B{}^D\phi_{(AC}{}^{GH}\phi_{FD)GH}\\+\kappa_C{}^D\phi_{(AB}{}^{GH}\phi_{FD)GH}+\kappa_F{}^D\phi_{(BC}{}^{GH}\phi_{AD)GH}.\label{Eq:UsefulIdentity1}
 \end{multline}
Using the last equation, along with the irreducible
decomposition
\begin{align*}
\phi_{ABCD} \phi_{FGH}{}^{D} &= \tfrac{1}{24} \phi_{DLMP} \phi^{DLMP}
\epsilon_{AH} \epsilon_{BG} \epsilon_{CF} + \tfrac{1}{24} \phi_{DLMP}
\phi^{DLMP} \epsilon_{AG} \epsilon_{BH} \epsilon_{CF} \\ & +
\tfrac{1}{24} \phi_{DLMP} \phi^{DLMP} \epsilon_{AH} \epsilon_{BF}
\epsilon_{CG} + \tfrac{1}{24} \phi_{DLMP} \phi^{DLMP} \epsilon_{AF}
\epsilon_{BH} \epsilon_{CG} \\ &+ \tfrac{1}{24} \phi_{DLMP}
\phi^{DLMP} \epsilon_{AG} \epsilon_{BF} \epsilon_{CH} + \tfrac{1}{24}
\phi_{DLMP} \phi^{DLMP} \epsilon_{AF} \epsilon_{BG} \epsilon_{CH}
\\ &+ \tfrac{1}{6} \epsilon_{CH} \phi_{(AB}{}^{DL}\phi_{FG)DL} +
\tfrac{1}{6} \epsilon_{CG} \phi_{(AB}{}^{DL}\phi_{FH)DL} +
\tfrac{1}{6} \epsilon_{CF} \phi_{(AB}{}^{DL}\phi_{GH)DL}\\ & +
\tfrac{1}{6} \epsilon_{BH} \phi_{(AC}{}^{DL}\phi_{FG)DL} +
\tfrac{1}{6} \epsilon_{BG} \phi_{(AC}{}^{DL}\phi_{FH)DL} +
\tfrac{1}{6} \epsilon_{BF} \phi_{(AC}{}^{DL}\phi_{GH)DL}\\ &+
\tfrac{1}{6} \epsilon_{AH} \phi_{(BC}{}^{DL}\phi_{FG)DL} +
\tfrac{1}{6} \epsilon_{AG} \phi_{(BC}{}^{DL}\phi_{FH)DL} +
\tfrac{1}{6} \epsilon_{AF} \phi_{(BC}{}^{DL}\phi_{GH)DL},
\end{align*}
we derive the identity
\begin{align}
     \kappa^{DG}\phi_{(ABC}{}^H\phi_{F)HDG}&=2\phi_{(AB}{}^{GH}B_{CF)GH}. \label{Eq:UsefulIdentity2}
\end{align}
Using the definition of the Buchdahl zero quantity
\eqref{KS_zero_quantities1}, the CFEs for $\bm\phi$
in its first and second order form, namely equations
\eqref{RescaledWeylEquationDisplayed} and \eqref{Wave_eq_CFE_Weyl},
and using the decomposition \eqref{Eq:DecompGradKS} we get
\begin{multline} 
\square B_{ABCD} = \tfrac{2}{3}  \nabla_{\bm\xi}\phi_{ABCD} + 8 \Lambda B_{ABCD} -   \phi_{(ABC}{}^{F}\mathcal{Q}_{D)F} + \tfrac{2}{3} H^{A'}{}_{(A}{}^{FG}\nabla_{|FA'|}\phi_{BCD)G} \\ - 6 \Xi \kappa_{(A}{}^{F}\phi_{BC}{}^{GH}\phi_{D)FGH} -  \Xi \kappa^{FG}\phi_{(ABC}{}^{H}\phi_{D)FGH}.
\end{multline}
where $\nabla_{\bm\xi} := \xi^{AA'}\nabla_{AA'}$. Substituting the above identities
\eqref{Eq:UsefulIdentity1}--\eqref{Eq:UsefulIdentity2}, we can derive
the following alternative (non-homogeneous) wave equation for
$B_{ABCD}$:
\begin{equation}
    \square B_{ABCD} = \tfrac{2}{3}\nabla_{\bm\xi}\phi_{ABCD} +
    8\Lambda B_{ABCD} - 14\Xi \phi_{(AB}{}^{FG}B_{CD)FG} +
    \tfrac{2}{3}(\nabla_{FA'}\phi_{G(ABC})H^{A'}{}_{D)}{}^{FG}. \label{Eq:SecondWaveEqForB}
\end{equation}
%%%%%%
%%\mnotex{EG: moved remarks to the end of the section to make the text better flow}
%%%%%%

%% \begin{remark}
%%   \emph{
%%   Notice that the collineation condition \eqref{Collineation} has some
%%   resemblance with the reduced conformal Killing vector initial data
%%   equations derived in \cite{Paetz14a} :
%%   \[
%% \mathcal{L}_{\mathring{\bmX}}d_{ij} +
%% \frac{1}{3}d_{ij}D_{k}\mathring{X}^k=0,
%% \]
%%  \mnotex{We can ommit reminding the reader of Paetz result if this represents puting a trap for
%%     ourselves. The referee could asks for the relation between the two conditions... to be discussed}
%%  \mnotex{JW: I think it's worth exploring, but presumably the $X$ here is a $3-$vector. How does it relate to our $X$.
%%    Also the Lie derivative the one intrinsic to the hypersurface?
%%    EG: yes in principle but let's simplify things and ommit this point for the moment}   
%%   where $\mathring{X}^a$ is a conformal Killing vector field on
%%   $(\mathscr{I},\bmh)$ where $\bmh$ denotes the induced metric by
%%   $\bmg$ on $\mathscr{I}$ with Levi-Civita connection $D$. Here,
%%   $d_{ij}$ represents the initial data for the electric part of the
%%   rescaled Weyl tensor.  However, as it will be shown in Section
%%   \ref{IntrinsicCKSID}
%%   the condition $\nabla_{\bm\xi}\phi_{ABCD}|_\mathcal{S}=0$ is
%%   trivially satisfied if $\bmH|_{\mathcal{S}}=0$ and
%%   $\bmB|_{\mathcal{S}}=0$. Hence our conformal Killing spinor initial
%%   data equations impose only algebraic restrictions on $\phi_{ABCD}$
%%   instead of differential ones.
%%   }
\begin{remark}
  \emph{
Equation \eqref{Eq:SecondWaveEqForB} actually encodes the fact that, \emph{given a Killing spinor} $\kappa_{AB}$, the field $\xi_{AA'}$ is a collineation for the rescaled Weyl tensor ---i.e. that
\[  \mathcal{L}_{\bm\xi}\phi_{ABCD} := \nabla_{\bm\xi}\phi_{ABCD} + \phi_{F(ABC}\nabla_{D)A'}\xi^{FA'}=0,\] 
the definition\footnote{The Lie derivative does not extend to spinor fields, in general. See \cite{PenRin86} for a discussion.} of the ``Lie derivative" being derived from the spinorialised counterpart of
$\mathcal{L}_{\bm\xi}d_{abcd}$. Indeed, this fact follows from a straightforward calculation
\begin{align*}
    \mathcal{L}_{\bm\xi}\phi_{ABCD}&:= \nabla_{\bm\xi}\phi_{ABCD}
    + \phi_{F(ABC}\nabla_{D)A'}\xi^{FA'} \\ &=
    \nabla_{\bm\xi}\phi_{ABCD}-6\Lambda \kappa_{(D}{}^{F} \phi_{ABC)F}
    - \tfrac{3}{2} \Xi \ \kappa^{FG} \phi_{(ABC}{}^{H} \phi_{D)FGH} +
    \tfrac{1}{4} \phi_{F(ABC} \ Q_{D)}{}^{F}\\ &=
    \nabla_{\bm\xi}\phi_{ABCD}-6\Lambda B_{ABCD} - 3\Xi
    \phi_{(AB}{}^{FG}B_{CD)FG} + \tfrac{1}{4}\phi_{F(ABC}Q_{D)}{}^F,
\end{align*}
where we are using \eqref{Eq:DecompGradXi} along with the identity
\eqref{Eq:UsefulIdentity2}. Given a Killing spinor $\kappa_{AB}$, the resulting zero quantities vanish:
$B_{ABCD}=H_{A'ABC}=Q_{AB}=0$, and hence it follows from
\eqref{Eq:SecondWaveEqForB} that
\begin{equation}\mathcal{L}_{\bm\xi}\phi_{ABCD} = \nabla_{\bm\xi}\phi_{ABCD} =0. \label{Collineation}
\end{equation}
Thus, a sub-product of our analysis is that if $(\mathcal{M},\bmg)$
admits a Killing spinor, then the Weyl-collineation condition
\eqref{Collineation} is satisfied.  This is not trivial since the
vector $\xi_{AA'}$ is not, in general, a
conformal Killing vector. In contrast,
for the physical spacetime it is clear that this condition holds 
since in that case $\tilde{\xi}_{AA'}$ is a Killing vector and thus
$\mathcal{L}_{\bm{\tilde{\xi}}}C_{abcd}=0$, trivially.\mnotex{Should be a tilde over the C?}
}
\end{remark}
%\begin{remark}\mnotex{Is this the right place for this remark? It kind of disrupts the flow, I think.} \emph{
%  If $\tilde{\kappa}_{AB}$ is a Killing spinor in the the physical
%  spacetime $(\tilde{\mathcal{M}},\tilde{\bmg})$ then $\tilde{X}_{AA'} :=
%  \nabla_A{}^{Q'}\tilde{\kappa}_{BA'}$  is a Killing
%  vector in the physical spacetime.  It is direct to check that
%  $X_{AA'}=\Xi^2 \tilde{X}_{AA'}$ is a conformal Killing vector in the
%  unphysical spacetime $(\mathcal{M},\bmg)$ with
%  $\bmg=\Xi^2\tilde{\bmg}$.  Hence, given that
%  $\kappa_{AB}=\Xi^{2}\tilde{\kappa}_{AB}$ a calculation using the
%  conformal transformation rules for the connection gives the following expression for the 
%  conformal Killing vector on the unphysical spacetime:
%\[
%X_{AA'}=\Xi \xi_{AA'} - 3 \kappa_{AQ}\nabla_{A'}{}^{Q}\Xi.
%\]
%Indeed, one finds
%\[\nabla_{AA'}X_{BB'}+\nabla_{BB'}X_{AA'} - \tfrac{1}{2}\epsilon_{AB}\epsilon_{A'B'}\nabla^{CC'}X_{CC'} = 2(\nabla_{C(A'}\Xi)H_{B')AB}{}^C - \Xi \nabla^C{}_{(A'}H_{B')ABC} \]
%by a straightforward calculation. 
%}
%\end{remark}

%\begin{remark}
%  \emph{
%  Notice that the collineation condition \eqref{Collineation} has some
 % resemblance with the reduced conformal Killing vector initial data
 % equations derived in \cite{Paetz14a} :
%  \[
% \mathcal{L}_{\mathring{\bmX}}d_{ij} +
% \frac{1}{3}d_{ij}D_{k}\mathring{X}^k=0,
% \]
% \mnotex{We can ommit reminding the reader of Paetz result if this represents puting a trap for
  %  ourselves. The referee could asks for the relation between the two conditions... to be discussed}
% \mnotex{JW: I think it's worth exploring, but presumably the $X$ here is a $3-$vector. How does it relate to our $X$. Also the Lie derivative the one intrinsic to the hypersurface?}   
 % where $\mathring{X}^a$ is a conformal Killing vector field on
 % $(\mathscr{I},\bmh)$ where $\bmh$ denotes the induced metric by
 % $\bmg$ on $\mathscr{I}$ with Levi-Civita connection $D$. Here,
 % $d_{ij}$ represents the initial data for the electric part of the
 % rescaled Weyl tensor.  However, as it will be shown in Section
 % \ref{IntrinsicCKSID}
 % the condition $\nabla_{\bm\xi}\phi_{ABCD}|_\mathcal{S}=0$ is
 % trivially satisfied if $\bmH|_{\mathcal{S}}=0$ and
 % $\bmB|_{\mathcal{S}}=0$. Hence our conformal Killing spinor initial
 % data equations impose only algebraic restrictions on $\phi_{ABCD}$
 % instead of differential ones.
 % }
%%  The latter condition can be strengthen assuming compactedness of
%%   $\mathscr{I}$ and that $(\mathscr{I},\bmh)$ is not conformal to the
%%   standard 3-sphere $(\mathcal{S}^3, \bar{\bmh})$ to
%% \[
%% \mathcal{L}_{\mathring{\bmX}}d_{ab} =0
%% \]
% \end{remark}

At this point we note that there are no derivatives of zero quantities
appearing on the right-hand-side of \eqref{Eq:SecondWaveEqForB}. This,
combined with the conformal field equation satisfied by
$\bm\phi$ as given in equation \eqref{RescaledWeylEquationDisplayed}
suggests
that by applying $\nabla^A{}_{A'}$ to equation
\eqref{Eq:SecondWaveEqForB} we may be able to derive a wave equation
for $F_{A'BCD}$ with the desired properties, namely being homogeneous
in $(\bmH, \bmH, \bmF)$ and their first derivatives (apart from the
D'Alembertian term). We will see that this strategy does indeed work;
the difficult part is in deriving a suitable expression for
$\nabla^A{}_{A'}\nabla_{\bm\xi}\phi_{ABCD}$ in terms of the zero
quantities.
%%\\
To do so, first, we note some useful identities.  From the definition
of the zero-quantities $\bmH$, $\bmB$ and the auxiliary spinor
$\bm\xi$ the following identity can be derived
\begin{align}
    \kappa_{A}{}^{F} \nabla_{FF'}\phi_{BCDG} &=\kappa_{A}{}^{F}
    \nabla_{DF'}\phi_{BCGF} \nonumber\\ &= \tfrac{1}{3} \xi_{AF'}
    \phi_{BCDG} - \tfrac{1}{3} \xi^{F}{}_{F'} \phi_{BCGF}
    \epsilon_{AD} - \tfrac{1}{3}
    \xi^{F}{}_{F'}\phi_{(BC|DF}\epsilon_{A|G)} \nonumber \\ & -
    \tfrac{2}{3} \xi^{F}{}_{F'}\phi_{(B|D|C|F}\epsilon_{A|G)} +
    \epsilon_{AD} F_{F'BCG} + \nabla_{(B|F'}B_{A|CG)D} + \tfrac{1}{3}
    \phi_{(BC|D}{}^{F}H_{F'A|G)F} \nonumber \\& + 2
    \epsilon_{A(B}F_{|F'\vert CG)D} - \tfrac{1}{3}
    \phi_{(BC}{}^{FH}H_{|F'|G)FH}\epsilon_{AD} - \tfrac{1}{6}
    \phi_{(BC}{}^{FH}H_{|F'DFH}\epsilon_{A|G)} \nonumber\\ & -
    \tfrac{1}{3} \phi_{(B|D}{}^{FH}H_{F'|C|FH}\epsilon_{A|G)} -
    \tfrac{1}{6}
    \phi_{D(B}{}^{FH}H_{|F'|C|FH}\epsilon_{A|G)}\label{Eq:MiscIdentity2}
\end{align}
from which follows 
% \begin{multline}
%     \kappa^{AD} \phi_{AD}{}^{GH} \nabla_{HA'}\phi_{BCFG} =  4 \xi^{A}{}_{A'}\phi_{(BC}{}^{DG}\phi_{F)ADG} + \tfrac{1}{2} \phi_{ADGH} \phi^{ADGH} H_{A'BCF} \\ + 4 B_{(BC}{}^{AD}\nabla^{G}{}_{|A'|}\phi_{F)ADG} - 4 B_{(B}{}^{ADG}\nabla_{|AA'|}\phi_{CF)DG} \\- 8 \phi_{(BC}{}^{AD}\nabla^{G}{}_{|A'|}B_{F)ADG} - 4 \phi_{(B}{}^{ADG}\nabla_{|AA'|}B_{CF)DG}  \\-  \tfrac{1}{3} \phi_{(BC}{}^{AD}\phi_{|AD}{}^{GH}H_{A'|F)GH} -  \tfrac{2}{3} \phi_{(B}{}^{ADG}\phi_{C|AD}{}^{H}H_{A'|F)GH}
% \end{multline}
\begin{align}
    \kappa^{AD} \phi_{AD}{}^{GH} \nabla_{HA'}\phi_{BCFG} &= 4
    \xi^{A}{}_{A'}\phi_{(BC}{}^{DG}\phi_{F)ADG} + \tfrac{1}{2}
    \phi_{ADGH} \phi^{ADGH} H_{A'BCF} \nonumber\\ & - 4
    B_{(B}{}^{ADG}\nabla_{|AA'|}\phi_{CF)DG} - 8
    \phi_{(BC}{}^{AD}F_{|A'|F)ADG}\nonumber \\ & - 4
    \phi_{(B}{}^{ADG}\nabla_{|AA'|}B_{CF)DG} - \tfrac{1}{3}
    \phi_{(BC}{}^{AD}\phi_{|AD}{}^{GH}H_{A'|F)GH}\nonumber \\ & -
    \tfrac{2}{3}
    \phi_{(B}{}^{ADG}\phi_{C|AD}{}^{H}H_{A'|F)GH} \label{Eq:MiscIdentity3}
\end{align}
---see Appendix \ref{Appendix_A} for details.
\\

We return now to the task of expressing $
\nabla^A{}_{A'}\nabla_{\bm\xi}\phi_{ABCD}$ in terms of
zero-quantities. First, commuting derivatives and using equation
\eqref{RescaledWeylEquationDisplayed} gives
\mnotex{Problems with indices all over the place, here!}
\begin{align}
    \nabla^A{}_{A'}\nabla_{\bm\xi}\phi_{ABCD} &=
    \nabla_{\bm\xi}(\nabla^A{}_{A'}\phi_{ABCD}) +
    (\nabla^A{}_{A'}\xi^{FF'})\nabla_{FF'} \phi_{ABCD} +
    \xi^{FF'}\left[\nabla^A{}_{A'},
      \nabla_{FF'}\right]\phi_{ABCD}\nonumber\\ & =
    (\nabla^A{}_{A'}\xi^{FF'})\nabla_{FF'} \phi_{ABCD} +
    \xi^{FF'}\left[\nabla^A{}_{A'}, \nabla_{FF'}\right]\phi_{ABCD}.
\end{align}
Then, using equations \eqref{Eq:DecompGradXi},
\eqref{RescaledWeylEquationDisplayed}
and expanding the commutator, renders
\begin{align*}
\nabla^A{}_{A'}\nabla_{\bm\xi}\phi_{ABCD} & = 6 \Lambda \xi^{D}{}_{A'} \phi_{ABCD} -  \xi^{DF'} \Phi_{D}{}^{F}{}_{A'F'} \phi_{ABCF} -  3\xi^{DF'} \Phi_{(A}{}^{F}{}_{\vert A'F'\vert} \phi_{BC)DF}  \nonumber\\
    &\quad -  3\Xi\xi^{D}{}_{A'}  \phi_{DFG(A} \phi_{BC)}{}^{FG} - 3\Lambda \kappa^{DF} \nabla_{FA'}\phi_{ABCD} + \tfrac{1}{4} \mathcal{Q}^{DF} \nabla_{FA'}\phi_{ABCD}  \nonumber\\ 
    &\quad -  \tfrac{3}{2} \kappa^{DF} \Phi_{D}{}^{G}{}_{A'}{}^{F'} \nabla_{FF'}\phi_{ABCG} -  \tfrac{3}{2} \kappa^{DF} \Phi_{D}{}^{G}{}_{A'}{}^{F'} \nabla_{GF'}\phi_{ABCF} \nonumber\\
    &\quad + \tfrac{3}{4} \Xi \kappa^{DF} \phi_{DF}{}^{GH} \nabla_{HA'}\phi_{ABCG} - \tfrac{1}{2}( \nabla_{FF'}\phi_{ABCD}) \nabla_{Q(A'}H_{F')F}{}^{DQ}.
\end{align*}
Finally, using equations \eqref{Eq:MiscIdentity2}--\eqref{Eq:MiscIdentity3}, one obtains
\begin{align}
\nabla^A{}_{A'}\nabla_{\bm\xi}\phi_{ABCD}  &= -3 \Phi _{A}{}^{D}{}_{A'}{}^{F'} F_{F'BCD}  - \tfrac{3}{8} \Xi  \phi _{DFGH} \phi ^{DFGH} H_{A'ABC} + \Lambda  \phi _{BCDF} H_{A'A}{}^{DF} \nonumber\\ 
    & \quad- \tfrac{1}{8} Q^{DF} \nabla_{FA'}\phi _{ABCD}   - \tfrac{3}{2} \Phi ^{DF}{}_{A'}{}^{F'} \nabla_{FF'}B_{ABCD} + \tfrac{1}{4} (\nabla_{DA'}H^{F'DFG}) \nabla_{GF'}\phi _{ABCF} \nonumber\\
    &\quad + \tfrac{1}{4} (\nabla_{D}{}^{F'}H_{A'}{}^{DFG}) \nabla_{GF'}\phi _{ABCF}  + 12 \Lambda  F_{A'ABC} +6\Xi \nabla^G{}_{A'}\left( B_{(AB}{}^{DF}\phi_{CG)DF}\right) \nonumber \\
    &\quad - \tfrac{9}{2} \Phi _{(B}{}^{D}{}_{|A'}{}^{F'}F_{F'A|CD)}  - \tfrac{3}{2} \Phi ^{DF}{}_{A'}{}^{F'}\nabla_{(B|F'}B_{A|CD)F} + \tfrac{1}{4} \Xi  \phi _{(BC}{}^{GH}\phi _{DF)GH} H_{A'A}{}^{DF}  \nonumber\\
    &\quad+ 2 \Lambda  \phi_{DF(AB}H_{\vert A'\vert C)}{}^{DF}+ 3 \Xi  \phi _{(BC}{}^{DF}F_{|A'A|D)F} + 6 \Xi  \phi _{A(B}{}^{DF}F_{|A'|CD)F} \nonumber \\
    &\quad  + \tfrac{3}{8} \Phi _{(B}{}^{D}{}_{|A'|}{}^{F'}\phi _{CD)}{}^{FG}H_{F'AFG} + \tfrac{9}{8} \Phi _{(B}{}^{D}{}_{|A'}{}^{F'}\phi _{A|C}{}^{FG}H_{|F'|D)FG} \nonumber\\
    &\quad + \Phi _{A}{}^{D}{}_{A'}{}^{F'}\phi _{(BC}{}^{FG}H_{|F'|D)FG}  - \tfrac{1}{2} \Phi ^{DF}{}_{A'}{}^{F'}\phi _{A(BC}{}^{G}H_{|F'|D)FG} \nonumber \\
    &\quad  - \tfrac{1}{2} \Phi ^{DF}{}_{A'}{}^{F'}\phi _{A(B|D}{}^{G}H_{F'|CF)G}+ \tfrac{1}{6}\Xi H_{A'A}{}^{DF}\phi_{(BC}{}^{GH}\phi_{DF)GH} \nonumber\\
    &\quad + \tfrac{1}{6}\Xi H_{A'B}{}^{DF}\phi_{(CA}{}^{GH}\phi_{DF)GH} + \tfrac{1}{6}\Xi H_{A'C}{}^{DF}\phi_{(AB}{}^{GH}\phi_{DF)GH}. \label{Eq:CollineationIdentity}
\end{align}
Note that the final expression is homogeneous in the zero quantities
$(\bmH, \bmB,\bmF)$ and their first derivatives.\\

We are now in a position to derive the remaining wave equation for
$F_{A'BCD}$.
\begin{align*}
    \square F_{A'BCD} &= \square (\nabla^A{}_{A'}B_{ABCD}) \\
    &= \left[\square, \nabla^A{}_{A'}\right] B_{ABCD} + \nabla^A{}_{A'}\square B_{ABCD} \\
    &= \left[\square, \nabla^A{}_{A'}\right] B_{ABCD} + \tfrac{2}{3}\nabla^A{}_{A'}\nabla_{\bm\xi}\phi_{ABCD} \\
    &\quad+ \nabla^A{}_{A'}\left(8\Lambda B_{ABCD} - 14\Xi \phi_{(AB}{}^{FG}B_{CD)FG} + \tfrac{2}{3}(\nabla_{FA'}\phi_{G(ABC})H^{A'}{}_{D)}{}^{FG}\right)\\
    &=   \tfrac{2}{3}\nabla^A{}_{A'}\nabla_{\bm\xi}\phi_{ABCD} -6 \Lambda F_{A'BCD} - 6 \Phi_{(B}{}^{A}{}_{\vert A'}{}^{B'} F_{B'\vert CD)A} - 9 B_{BCDA} \nabla^{A}{}_{A'}\Lambda \\
    &\quad + 3B_{(BC}{}^{FG} \phi_{D)AFG} \nabla^{A}{}_{A'}\Xi  + 3B_{AF(BC} \nabla^{FB'}\Phi_{D)}{}^{A}{}_{A'B'}  - 6 \Xi \phi_{(B}{}^{AFG} \nabla_{\vert GA'\vert}B_{CD)AF} \\
    &\quad + \nabla^A{}_{A'}\left(8\Lambda B_{ABCD} - 14\Xi \phi_{(AB}{}^{FG}B_{CD)FG} + \tfrac{2}{3}(\nabla_{FA'}\phi_{G(ABC})H^{A'}{}_{D)}{}^{FG}\right),
\end{align*}
where we are using \eqref{Eq:SecondWaveEqForB} in the third line and
in the fourth we are expanding out the commutator and using the
Bianchi identites. Substituting equation \eqref{Eq:CollineationIdentity}, we
obtain a homogeneous expression in $(\bmH, \bmB,\bmF)$ and their first
derivatives, as required.
\\

With this closed system of homogeneous wave equations at hand,
a direct application of the uniqueness property for solutions of wave equations
of Theorem \ref{TheoremHomogeneousWave} gives the following
\begin{proposition}\label{Prop:Propagation_KS}
  Given initial data for the conformal field equations on $\mathcal{U}\subseteq\mathcal{S}$
  where $\mathcal{S}$ is a spacelike
hypersurface $\mathcal{S}$ with normal vector $\tau^{AA'}$, and
associated normal derivative $\nabla_{\bm\tau} :=
\tau^{AA'}\nabla_{AA'}$, the corresponding spacetime development
admits a (valence-2) Killing spinor in $\mathcal{D}^{+}(\mathcal{U})$
%%%%%%%%%%
%---the future domain of dependence of $\mathcal{U}$---
%%%%%%%%%
if and only if
\begin{subequations}
\begin{eqnarray}
  && H_{A'ABC}=B_{ABCD}=F_{A'ABC}=0,\label{Eq:KSInitialCondition1}\\ &&
   \nabla_{\bm\tau} H_{A'ABC}=\nabla_{\bm\tau} B_{ABCD}= \nabla_{\bm\tau} F_{A'ABC}=0 \label{Eq:KSInitialCondition2}.
\end{eqnarray}
\end{subequations}
 hold on $\mathcal{U}$.
\end{proposition}
To summarise: we have found necessary and sufficient conditions,
namely \eqref{Eq:KSInitialCondition1}--\eqref{Eq:KSInitialCondition2},
defined on $\mathcal{U}\subset \mathcal{S}$, for a given Killing
spinor candidate, $\kappa_{AB}$, to give rise to a Killing spinor when
evolved according to the Killing spinor candidate wave equation encoded in $Q_{AB}=0$
---see equation \eqref{Eq:WaveForKS}.



%\begin{remark}
  %%%%
  %%%% \mnotex{Check that the signs are correct. EG: it is}
  %%%%
%  \emph{
 % In general, the Lie derivative (along non-conformal Killing vectors) 
%  does not extend to spinor fields   ---see
%  \cite{PenRin86} for a discussion. Nontheless, one can
%  define the following shorthand
% \[ \mathcal{L}_{\bm\xi}\phi_{ABCD} := \nabla_{\bm\xi}\phi_{ABCD} + \phi_{F(ABC}\nabla_{D)A'}\xi^{FA'},\]
% which is simply the spinorialised counterpart of
% $\mathcal{L}_{\bm\xi}d_{abcd}$. One can then show the following
% relation between this shorthand and the Killing spinor zero-quantities,
% \begin{align*}
%    \mathcal{L}_{\bm\xi}\phi_{ABCD}&:= \nabla_{\bm\xi}\phi_{ABCD}
%    + \phi_{F(ABC}\nabla_{D)A'}\xi^{FA'} \\ &=
%    \nabla_{\bm\xi}\phi_{ABCD}-6\Lambda \kappa_{(D}{}^{F} \phi_{ABC)F}
%    - \tfrac{3}{2} \Xi \ \kappa^{FG} \phi_{(ABC}{}^{H} \phi_{D)FGH} +
%    \tfrac{1}{4} \phi_{F(ABC} \ Q_{D)}{}^{F}\\ &=
%    \nabla_{\bm\xi}\phi_{ABCD}-6\Lambda B_{ABCD} - 3\Xi
%    \phi_{(AB}{}^{FG}B_{CD)FG} + \tfrac{1}{4}\phi_{F(ABC}Q_{D)}{}^F
%\end{align*}
%where we are using equation \eqref{Eq:DecompGradXi}, along with the identity
%\eqref{Eq:UsefulIdentity2}. Therefore, given a Killing spinor $\kappa_{AB}$,
%the resulting zero quantities vanish:
%$B_{ABCD}=H_{A'ABC}=Q_{AB}=0$, and hence it follows from
%\eqref{Eq:SecondWaveEqForB} that
%\begin{equation}\mathcal{L}_{\bm\xi}\phi_{ABCD} = \nabla_{\bm\xi}\phi_{ABCD} =0. \label{Collineation}
%\end{equation}
%Thus, a sub-product of our analysis is that if $(\mathcal{M},\bmg)$
%admits a Killing spinor, then the Weyl-collineation condition
%\eqref{Collineation} is satisfied.  This is not trivial since the
%vector $\xi_{AA'}$ is not, in general, a
%conformal Killing vector. In contrast,
%for the physical spacetime is clear that this condition holds 
%since in that case $\tilde{\xi}_{AA'}$ is a Killing vector and thus
%$\mathcal{L}_{\bm{\tilde{\xi}}}C_{abcd}=0$, trivially.
%}
%\end{remark}

\subsection{Intrinsic (``CKSID") conditions}
\label{IntrinsicCKSID}

In this section, we aim to reduce \eqref{Eq:KSInitialCondition1}--\eqref{Eq:KSInitialCondition2} to a set of intrinsic conditions ---that is to say, conditions on $\kappa_{AB}$ that are computable at the level of an initial data manifold. Clearly the condition 
\begin{equation}
B_{ABCD}\equiv \kappa_{(A}{}^F\phi_{BCD)F}=0. \label{Eq:VanishingBuchdahl}
\end{equation}
is already intrinsic ---see Remark \ref{G-C-M-Remark}. 
%\begin{subequations}
%\begin{eqnarray}
 %     && \tau_{(A}{}^{A'}H_{\vert A'\vert BCD)} :=  \mathcal{D}_{(AB}\kappa_{CD)}=0, \label{Eq:SpatialKS}\\
 %     && B_{ABCD}:= \kappa_{(A}{}^F\phi_{BCD)F} = 0 \label{Eq:VanishingBuchdahl}
%\end{eqnarray}
%\end{subequations}
%on $\mathcal{U}$. 
The condition $H_{A'ABC}\big\vert_{\mathcal{U}}=0$, on the other hand, implies the following intrinsic condition
\begin{equation}
 \tau_{(A}{}^{A'}H_{\vert A'\vert BCD)} :=  \mathcal{D}_{(AB}\kappa_{CD)}=0. \label{Eq:SpatialKS}
\end{equation}
We evolve the Killing spinor candidate according to \eqref{Eq:WaveForKS} and the initial condition
\begin{equation}
     \nabla_{\bm\tau} \kappa_{BC} = - \xi_{BC}\label{InitialCondition}
\end{equation}
on $\mathcal{U}$, with $\xi_{AB}:=\mathcal{D}_{(A}{}^C\kappa_{B)C}$. This can be easily shown to be equivalent to $\tau^{AA'}H_{A'ABC}=0$, and so \eqref{Eq:SpatialKS} and \eqref{InitialCondition} together encode $H_{A'ABC}\big\vert_{\mathcal{S}}=0$. Decomposing the covariant derivative, we have on $\mathcal{S}$
\begin{align}
\xi_{AB'} = - \tau_{B}{}^{A'} \tau^{B}{}_{B'}\xi_{AA'}
&=\tfrac{1}{2}\tau^{B}{}_{B'}\nabla_{\bm\tau}\kappa_{AB} +
\ \tau^{B}{}_{B'}
\mathcal{D}_{BC}\kappa_{A}{}^{C}. \nonumber\\ &=\tfrac{1}{2}\tau^{B}{}_{B'}\nabla_{\bm\tau}\kappa_{AB}
- \tau^B{}_{B'}\xi_{AB} + \xi\tau_{AB'} \nonumber\\ &= -\tfrac{3}{2}
\tau^B{}_{B'}\xi_{AB} +
\xi\tau_{AB'} \label{DecompXiOnSUsingInitialCondition},
\end{align}
with $\xi:=\mathcal{D}^{AB}\kappa_{AB}$, and where in the last line we
are using equation \eqref{InitialCondition}. Starting from
\eqref{IrrDecompCurlOfH}, and performing the decomposition of the
covariant derivative, we get
\[ \nabla_{\bm\tau} H_{B'ABC} + 2 \tau^{D}{}_{B'} \tau^{FA'} \mathcal{D}_{DF}H_{A'ABC} = -12 \Xi \tau^{D}{}_{B'}B_{ABCD} + \tfrac{3}{2} \tau_{(A|B'|}Q_{BC)}.\]
At this point we see that if
\eqref{Eq:VanishingBuchdahl}--\eqref{InitialCondition} hold, then
$\nabla_{\bm\tau} H_{B'ABC}\big\vert_{\mathcal{S}}=0$ also.

\medskip

Now we move onto analysing the constraints imposed by $B_{ABCD}$ and
its time derivative on $\mathcal{S}$.  A similar computation to the
twistor case yields
\begin{equation}
    \nabla_{\bm\tau} B_{ABCD} = 2
    \kappa_{(A}{}^{F}\mathcal{D}_{B}{}^{G}\phi_{CD)FG} +
    \phi_{(ABC}{}^{F}\xi_{D)F}.\label{EvolutionForBuchdahl}
\end{equation}
Note that the quantity on the right-hand-side is intrinsic to
$\mathcal{S}$; for the meantime, we append
\begin{equation}
2 \kappa_{(A}{}^{F}\mathcal{D}_{B}{}^{G}\phi_{CD)FG} + \phi_{(ABC}{}^{F}\xi_{D)F}=0\label{IntrinsicCondition3}
\end{equation}
to our list of intrinsic
conditions.
%%%%
%%%(for the meantime).
%%%
Consider now the quantity
$F_{A'ABC}$; recall that, by definition,
$F_{A'ABC}=\nabla^D{}_{A'}B_{ABCD}$, and so decomposing the covariant
derivative one obtains
\[F_{A'BCD} = \tfrac{1}{2} \tau^{A}{}_{A'} \nabla_{\bm\tau} B_{ABCD}  -  \tau^{F}{}_{A'} \mathcal{D}^{A}{}_{F}B_{ABCD}.\]
Hence, given the vanishing of $B_{ABCD}$ and Hence, given the vanishing of $B_{ABCD}$ and \eqref{IntrinsicCondition3}, we see that $F_{A'ABC}=0$ on
$\mathcal{S}$, as well.
%%%%%%%%%%%%%%%%
%%Thus, all that's left is to find intrinsic
%%%%%%%%%%%%%%
Thus it only remains to find the intrinsic 
conditions under which $\nabla_{\bm\tau}F_{A'ABC}=0$ on
$\mathcal{S}$.

\medskip
%%%%%%%%%%
%%Now,
%%%%%%%%
Combining the two wave equations for $B_{ABCD}$,
\eqref{Eq:FirstWaveEqForB} and \eqref{Eq:SecondWaveEqForB}, we see
that
\begin{multline}
     \nabla_{FA'}F^{A'}{}_{BCD} = \tfrac{1}{3} \nabla_{\xi}{}\phi_{BCDF} - 2 \Lambda B_{BCDF} + 3\Xi B_{(BC}{}^{AG}  \phi_{D)FAG}  - 7 \Xi B_{(BC}{}^{AG}\phi_{DF)AG} \\-  \tfrac{1}{2} \phi_{(BCD}{}^{A}\mathcal{Q}_{F)A} + \tfrac{1}{3} H^{A'}{}_{(B}{}^{AG}\nabla_{|AA'|}\phi_{CDF)G}\label{CurlOfFInTermsOfCollineation}. 
\end{multline}
Decomposing the covariant derivative, we get
\begin{multline}
\nabla_{\bm\tau} F_{B'BCD} + 2 \tau^{A}{}_{B'} \tau^{FA'} \mathcal{D}_{AF}F_{A'BCD} \\
= \tau^{A}{}_{B'}\left(\tfrac{2}{3} \nabla_{\xi}{}\phi_{BCDA}  - 4 \Lambda  B_{BCDA} + 6 \Xi  B_{(BC}{}^{FG}  \phi_{D)AFG}  \right. \\ \left.  - 14 \Xi  B_{(BC}{}^{AG}\phi_{DA)FG}  -   \phi_{(BCD}{}^{F}\mathcal{Q}_{A)F}   + \tfrac{2}{3}  H^{A'}{}_{(B}{}^{FG}\nabla_{|FA'|}\phi_{CDA)G}\right).
\end{multline}
Hence, if
$F_{A'ABC}\big\vert_{\mathcal{S}}=H_{A,ABC}\big\vert_{\mathcal{S}}=B_{ABCD}\big\vert_{\mathcal{S}}=0$,
then $\nabla_{\bm\tau}F_{A'ABC}\big\vert_{\mathcal{S}}=0$ if and only
if
\[ \nabla_{\bm\xi}\phi_{ABCD}\big\vert_{\mathcal{S}}=0.\]
Decomposing the covariant derivative, using the evolution equation $\bm\phi$
%%%%
%% for rescaled Weyl
%%%%
as given in equation \eqref{RescaledWeyl_evo_const}, along with equation
\eqref{DecompXiOnSUsingInitialCondition} we have
\[ \nabla_{\bm\xi}\phi_{ABCD}\big\vert_{\mathcal{S}} =  \xi \mathcal{D}_{(A}{}^{F}\phi_{BCD)F} + \tfrac{3}{2} \xi^{FG} \mathcal{D}_{FG}\phi_{ABCD}. \]
Hence, at this point then, we have reduced our initial conditions for the
zero quantities to the following set of intrinsic conditions:
\begin{subequations}
\begin{eqnarray}
    && \mathcal{D}_{(AB}\kappa_{CD)}=0,\\ &&
  \kappa_{(A}{}^F\phi_{BCD)F}=0,\\ && 2
  \kappa_{(A}{}^{F}\mathcal{D}_{B}{}^{G}\phi_{CD)FG} +
  \phi_{(ABC}{}^{F}\xi_{D)F} = 0,\label{RedundantCondition1}\\ &&
  \xi^{FG}\mathcal{D}_{FG}\phi_{ABCD} +
  \tfrac{2}{3}\xi\mathcal{D}_{(A}{}^F\phi_{BCD)F} =
  0. \label{RedundantCondition2}
\end{eqnarray}
\end{subequations}
Although equations \eqref{Raw_CSKID} encode
necessary and sufficient conditions for the existence of a
Killing spinor, the redundancy relations between these equations are established
in the following
\begin{proposition}\label{prop_remove_redundant_conditions}
Suppose that either $\kappa_{AB}\kappa^{AB}:= 0$ on $\mathcal{U}$ or
that $\kappa_{AB}\kappa^{AB}\neq 0$ everywhere on $\mathcal{U}$, then
conditions \eqref{RedundantCondition1} and \eqref{RedundantCondition2}
are redundant in the sense that they are automatically satisfied by
$\kappa_{AB}$ by virtue of the first two conditions.
\end{proposition}

The proof of this proposition requires decomposing the fields respect to a spin dyad
and considering the cases where $\bm\phi$ is of different Petrov types.
This is a long but direct calculation that is given in Appendix \ref{Sec:ProofOfProp3}.


\begin{remark}{\em 
Note that if it is not the case that $\kappa_{AB}\kappa^{AB}:= 0$ on
$\mathcal{U}$, then there must exist an open subset
$\tilde{\mathcal{U}}\subset\mathcal{U}$ on which
$\kappa_{AB}\kappa^{AB}$ is nowhere zero, since the vanishing of
$\kappa_{AB}\kappa^{AB}$ is a closed condition, by the assumed
continuity (in fact, differentiability) of the Killing spinor
candidate. Hence, the above proposition covers all eventualities, with
the small caveat that one may have to restrict $\mathcal{U}$.  }
\end{remark}

\noindent
Finally, putting together Propositions
\ref{Prop:Propagation_KS} and \ref{prop_remove_redundant_conditions}
gives the following


\begin{theorem}\label{Theorem_KS}
Consider an initial data set for the vacuum conformal Einstein
field equations, as encoded in the CFE zero-quantities
\eqref{Def_ConfFactor_CFE_zeroquant}--\eqref{Def_cons_CFE_zeroquant},
on a spacelike hypersurface $\mathcal{S}$ and let
$\mathcal{U}\subset\mathcal{S}$ denote an open set.
The development
of the initial data set will have a
(valence-2) Killing spinor in the domain of
dependence of $\mathcal{U}$ if and only if there exists $\kappa_{AB}$ defined over $\mathcal{U}$ satisfying\mnotex{Slight technicality concerning restriction of open set.}
\[ \mathcal{D}_{(AB}\kappa_{CD)}=0, \qquad \kappa_{(A}{}^F\phi_{BCD)F}=0\]
on $\mathcal{U}$. Given a solution to the above on $\mathcal{U}$, the Killing spinor is obtained by 
evolving the initial data $\kappa_{AB}$ according to the wave equation:
\begin{align} \label{Wave_eq_KS_theo}
\square \kappa _{AB} = -4 \Lambda \kappa _{AB} + \Xi \phi_{ABCD}\kappa^{CD}.
\end{align}
and subject to the initial condition
\begin{equation}
  \nabla_{\bm\tau} \kappa _{AB} = -\mathcal{D} _{(A}{}^C \kappa_{B)C}.
\end{equation}
\end{theorem}
\begin{proof}
Hello world\mnotex{Write-up of proof needed.}
\end{proof}

\begin{remark}{\em 
Note that we recover the conditions from \cite{BaeVal10a}, namely 
\[\tilde{\mathcal{D}}_{(AB}\tilde{\kappa}_{CD)}=0, \qquad \tilde{\kappa}_{(A}{}^F
\Psi_{BCD)F}=0, \]
when $\Xi\neq 0$ on $\mathcal{U}$ (i.e. when $\mathcal{U}\cap \mathscr{I} = \emptyset $). On the other hand, note that the condition $\kappa_{(A}{}^F\Psi_{BCD)F}=0$ would trivialise if one were to evaluate it at some $p\in \mathscr{I}$, since the existence of a regular compactification necessarily implies that $\Psi_{ABCD}=0$ on $\mathscr{I}$.
}
\end{remark}
\begin{remark}
  \emph{
  Since the fields comprising initial data for the CFEs contain the
  rescaled Weyl spinor, one can use the initial datum
  $\phi_{ABCD}|_{\mathcal{S}}$ to formally define the Petrov type of
  an initial data set for the CFEs. This can be expressed covariantly as $I^3 - 6J^2=0$, where $I,J$ are the complex-valued scalars
\begin{align*}
I &:= \phi_{ABCD}\phi^{ABCD} \equiv E_{ij}E^{ij} - B_{ij}B^{ij} + 2i E_{ij}B^{ij},\\
J &:= \phi_{AB}{}^{CD}\phi_{CD}{}^{FG}\phi_{FG}{}^{AB} \equiv (E_{i}{}^jE_j{}^kE_{k}{}^i - 3E_{i}{}^jB_j{}^kB_{k}{}^i) + i(3B_{i}{}^jE_j{}^kE_k{}^{i} - B_{i}{}^jB_j{}^kB_k{}^i),
\end{align*}
with $E_{ij}, B_{ij}$ denoting the electric and magnetic parts of the rescaled Weyl tensor, respectively. Type $N$ corresponds to the special case $I=J=0$. 
    }
\end{remark}

%%%%%%%%%%%%%%%%%%%%%%%%%%%%
%%%%%%%%% References %%%%%%
%%%%%%%% https://arxiv.org/pdf/2101.00856.pdf
%%%%%%%% https://arxiv.org/pdf/1010.2421.pdf
%%%%%%%% See page 107-108, section Killing spinors for type {22} vacuums of \cite{PenRin84}
%%%%%%%% Lemma 1 of https://link.springer.com/article/10.1007/BF01649445
%%%%%%%%%%%%%%%%%%%%%%%%%%%%%%

\begin{corollary}\label{Corollary:PetrovPropagation}
%Suppose that $\phi_{ABCD}$ is of Petrov type N on $\mathcal{U}\subset \mathcal{S}$, then $\phi_{ABCD}$ is of Petrov type N on an open\mnotex{is it open?} subset of the spacetime development, $\mathcal{D}^+(\mathcal{U})$,  as evidenced by the existence of a non-trivial twistor (i.e. valence-1 Killing spinor). 
Suppose that $\phi_{ABCD}$ is of Petrov type D on $\mathcal{U}\subset\mathcal{S}$ and suppose further that 
\begin{equation} 
\sigma\equiv - o^A o^B o^C\mathcal{D}_{AB}o_C=0 \quad \text{and}\quad \lambda\equiv \iota^A\iota^B\iota^C\mathcal{D}_{AB}\iota_C=0\label{ShearConditions}
\end{equation}
hold on $\mathcal{U}$, where $\lbrace \bmo, \bm\iota\rbrace$ is an adapted spin dyad in terms of which $\phi_{ABCD}=\phi o_{(A}o_B\iota_C\iota_{D)}$ for some field $\phi$.
Then, $\phi_{ABCD}$ is also of Petrov type D on an open subset of the spacetime development, $\mathcal{D}^+(\mathcal{U})$, as evidenced by the existence of a non-trivial (valence-2) Killing spinor satisfying $\kappa_{AB}\kappa^{AB}\neq 0$. 
\end{corollary}
\begin{proof}
%Suppose that $\phi_{ABCD}$ is of Petrov type N on $\mathcal{U}$, with $\lbrace \bmo, \bm\iota\rbrace$ an adapted spin dyad in terms of which $\phi_{ABCD} = \phi o_Ao_Bo_Co_D$ then as shown in \ref{TypeNCase} the 
%\medskip
Suppose that $\phi_{ABCD}$ is of Petrov type D on $\mathcal{U}$, with $\lbrace \bmo, \bm\iota\rbrace$ an adapted spin dyad in terms of which $\phi_{ABCD}=\phi o_{(A}o_B\iota_C\iota_{D)}$ for some field $\phi$. Define $\kappa_{AB}=\phi^{-1/3}o_{(A}\iota_{B)}$. The Buchdahl constraint is clearly satisfied. The equation $\mathcal{D}_{(AB}\kappa_{CD)}=0$ decomposes to give 
\begin{eqnarray*}
&& \sigma=0,\\
&& \mathcal{D}_{\bm0\bm0}\phi = 3\phi(\kappa - \tau), \\
&& \mathcal{D}_{\bm0\bm1}\phi = \tfrac{3}{2} \phi(\rho - \mu),\\
&& \mathcal{D}_{\bm1\bm1}\phi = 3\phi(\nu - \pi), \\
&& \lambda = 0.
\end{eqnarray*}
To see this, substitute $e^{\varkappa}=\phi^{-1/3}$ in \eqref{SpatialSenTypeD0000}--\eqref{SpatialSenTypeD1111}.
The first and last components are guaranteed by the assumption \eqref{ShearConditions}.
The remaining equations are precisely those implied by the constraint $\mathcal{D}^{AB}\phi_{ABCD}=0$, namely  \eqref{WeylConstraintTypeD}.
%\[ o^Ao^Bo^C\iota^D\mathcal{D}_{(AB}\kappa_{CD)}=o^Ao^B\iota^C\iota^D\mathcal{D}_{(AB}\kappa_{CD)}=o^A\iota^B\iota^C\iota^D\mathcal{D}_{(AB}\kappa_{CD)}=0\]
Hence, the Killing spinor initial data conditions are met by $\kappa_{AB}$ and propagating we obtain a Killing spinor in some open subset of $\mathcal{D}^+(\mathcal{U})$, according to Theorem \ref{Theorem_KS}. Finally, $I\neq 0$ on $\mathcal{U}$ we have by continuous dependence on data that $I\neq 0$ on...
\end{proof}

\begin{remark}{\em 
 While the constraint $\mathcal{D}^{AB}\phi_{ABCD}=0$ implies that the $\bm0\bm0\bm0\bm1$, $\bm0\bm0\bm1\bm1$ and $\bm0\bm1\bm1\bm1$-components of $\mathcal{D}_{(AB}\kappa_{CD)}=0$
  are all satisfied, as shown in the above proof, the extremal spin components of $\mathcal{D}_{(AB}\kappa_{CD)}=0$ are not guaranteed simply as a consequence of the above expression for $\kappa_{AB}$. Hence, the conditions of Theorem \ref{Theorem_KS} are more restrictive than simply assuming that $\bm\phi$ be of Petrov type D on $\mathcal{U}$. The remaining two extremal components are however implied by the $\bm0\bm0\bm0\bm0$ and $\bm1\bm1\bm1\bm1$-components of the \emph{evolution equation} $\nabla_{\bm\tau}\phi_{ABCD}=2\mathcal{D}_{(A}{}^F\phi_{BCD)F}$ \emph{if} one assumes that the Petrov type extends to the spacetime development, consistent with Remark \ref{Remark:DyadExpressionForKillingSpinorInTypeD}.
  }
\end{remark}


\section*{Conclusions}

In this article a \emph{conformal} version of the Killing spinor
initial data equations given in \cite{GarVal08c} are derived. By
conformal it is understood that $(\mathcal{M},\bmg)$ is conformally
related to a vacuum Einstein spacetime $(\tilde{\mathcal{M}},\tilde{\bmg})$.
%%%%%%%%%%%%%%%%%%%%%%%
%Consequently, we call these
%conditions the \emph{conformal Killing spinor initial data equations}.
%%%%%%%%%%%%%%
It is shown that the existence of a non-trivial solution of equations
\eqref{CSKID_1}-\eqref{CSKID_2} is a necessary and sufficient
condition for the existence of a Killing spinor on the development in
the unphysical spacetime $(\mathcal{M},\bmg)$ governed by Friedrich's
\emph{conformal Einstein field equations}.
%%%%%%%%%%%%%%%%%%%%
%These conditions are intrinsic to a
%spacelike hypersurface $\mathcal{S}\subset\mathcal{M}$.
%%%%%%%%%%%%%%%%%%%%
  The initial data equations derived in this article contain one differential condition and one
  algebraic condition.  The differential condition corresponds to the
  so-called \emph{spatial Killing spinor equation} while the algebraic
  condition is a restriction imposed by the \emph{Buchdahl constraint} on the initial
  hypersurface. This constraint can be interpreted as restricting the Petrov type
  of an initial data set for the conformal Einstein field equations.
  Although conditions \eqref{CSKID_1}-\eqref{CSKID_2} look formally
  identical to those derived for the physical spacetime
  $(\tilde{\mathcal{M}},\tilde{\bmg})$, since the Einstein field
  equations are not conformally invariant, the proof that these
  conditions also hold for the unphysical spacetime
  $(\mathcal{M},\bmg)$ is non-trivial.  Moreover, the raw expressions
  of proposition \ref{Prop:Propagation_KS} would suggest that one
  needs further conditions besides \eqref{CSKID_1}-\eqref{CSKID_2} to
  ensure the existence of a Killing spinor in $(\mathcal{M},\bmg)$. It
  is only after exploring the redundancy between the raw conditions
  \eqref{Raw_CSKID} that one obtains the reduced set
  \eqref{CSKID_1}-\eqref{CSKID_2}.

  \medskip

In the case where the conformal rescaling is
trivial, $\Xi = 1$, we recover the results of
\cite{BaeVal10b}. However, even for this case, the set of variables to
be propagated in the physical and the unphysical spacetimes
are different.  This difference can be traced back to the observation
that for a general Lorentzian manifold the vector
$\xi_{AA'}=\nabla^{B}{}_{A'}\kappa_{AB}$ is not a Killing vector.
Furthermore, even in the case where $(\mathcal{M},\bmg)$ satisfies the
conformal Einstein field equations, this vector does not correspond to
a conformal Killing vector as one could naively expect but rather to a
Weyl collineation which is a subproduct of the analysis of this
article. Naturally, once the existence of a
Killing spinor in the unphysical spacetime $(\mathcal{M},\bmg)$ is
established one can always construct, a posteriori, a conformal
Killing vector $X_{AA'}$ which corresponds to a Killing vector
$\tilde{\xi}_{AA'}$ of physical spacetime
$(\tilde{\mathcal{M}},\tilde{\bmg})$.
%%%%%%%%%%%%%%%%%%%%%
%% The first algebraic condition
%% corresponds to the restriction of the Buchdahl constraint on the
%% initial hypersurface and the second imposes restrictions on the Cotton
%% spinor of the initial data set. Moreover, it was shown that, in a spin
%% dyad adapted to the Killing spinor, these conditions can be used along
%% with the conformal Einstein field equations to show that certain
%% components (at least half of them) of the Cotton spinor $Y_{ABCA'}$
%% have to vanish on the initial hypersurface $\mathcal{S}$.
%%%%%%%%%%%%%%%%%%%
Notice that the conformal approach followed in this article ---i.e.,
use of the conformal Einstein field equations--- opens the possibility
to have $\mathcal{S}$ determined by $\Xi = 0$ so that it to
corresponds to the conformal boundary $\mathscr{I}$.  In other words,
it allows to identify \emph{asymptotic initial data} ---initial data for the
conformal Einstein field equations given on $\mathscr{I}$--- for de
Sitter-like spacetimes whose development will contain a Killing spinor
$\kappa_{AB}$.  Hence the analysis given in this article has
applicability in characterisations of the Kerr-de Sitter
spacetime, via the existence of Killing spinors at the (spacelike)
conformal boundary.
%%%%%%%%%%%%%%%%%%%%%%%%%
%% % the Cotton spinor will play
%% %a replant role. This is not unexpected since the conformal boundary of
%% %the Kerr-de Sitter spacetime is conformally flat ---see
%% %\cite{AshBonKes15a, Olz13}.  Therefore, the Cotton tensor associated
%% %with asymptotic initial data corresponding to the Kerr-de Sitter
%% %spacetime vanishes.
%%%%%%%%%%%%%%%%%%%%%

\medskip

Moreover, the applications of the core analysis of this article
are not restricted to the study of de Sitter-like spacetimes since the most delicate part of the analysis
consisted on finding a system of homogeneous wave equations for
$H_{A'ABC}$ and $B_{ABCD}$ and $F_{A'ABC}$ which are irrespective of
the causal nature of $\mathcal{S}$. For the spacelike case, the
uniqueness result for wave equations of Theorem
\ref{TheoremHomogeneousWave} ensures that if trivial initial data on
$\mathcal{S}$ is provided then $H_{A'ABC}$, $B_{ABCD}$ and $F_{A'ABC}$
vanish on the domain of dependence of the data which constitutes the
of the main technical tool to derive the main result.
Nonetheless, analogous theorems for the characteristic problem or
the initial boundary value problem could be used to obtain similar
conditions considering a null or timelike hypersurface as it has been done
some applications of the Killing vector initial data equations and the
physical Killing spinor initial data equations
---see \cite{Pae14a, ColRacVal18} and \cite{CarVal18}.
%%%%%%%%%%%%%%%%%%%%%%%%
%% In the case of a spacelike $\mathcal{S}$ This system of wave equations
%% in turn, leads to conditions \eqref{CSKID_1}-\eqref{CSKID_2}
%% Consequently, one could investigate the analogous
%% conditions to those derived in section \ref{IntrinsicCKSID}
%% considering a timelike or null hypersurface $\mathcal{S}$ instead.
%%%%%%%%%%%%%%%%%%%%%%%%
In the case of the characteristic problem, the conformal approach of
this article would allow to analyse the existence of Killing spinors
at the conformal boundary of an asymptotically flat spacetimes.  In the
case of a timelike hypersurface, the analogous conformal Killing
spinor initial data equations could be used in the analysis and
characterisation of anti-de Sitter like spacetimes.


\subsection*{Acknowledgements}

We would like to thank Juan A. Valiente Kroon, David Hilditch and Justin Feng for helpful discussions.
E. Gasper\'in acknowledges support from Consejo Nacional de
Ciencia y Tecnolog\'ia (Mexico) ---CONACyT studentship
494039/218141--- in the early stages of this work and from Fundaç\~ao
para a Ci\^encia e a Tecnologia (Portugal) ---FCT-2020.03845.CEECIND---
during its completion. J. L. Williams acknowledges support from the COST grant CA16104.  
\appendix

\section{Appendix}\label{Appendix_A}

Here we give the derivation of some of the key identities used in Section \ref{Sec:KSWaveEqs} and the proof of Proposition \ref{prop_remove_redundant_conditions} from Section \ref{IntrinsicCKSID}. 

\subsection{Spinorial identities}
\mnotex{Prepend equation numbers with ``A"}
Here we give the derivation of identities \eqref{Eq:MiscIdentity2} and \eqref{Eq:MiscIdentity3}. First, by expanding out the definition of $B_{ABCD}$,
\begin{align*}
\nabla^D{}_{G'}B_{ABFD} &= - \tfrac{1}{4} \phi_{ABFD} \nabla_{CG'}\kappa^{CD}  -  \tfrac{3}{4} \kappa_{(A}{}^{C} \nabla_{\vert DG'\vert}\phi_{BF)C}{}^{D} -  \tfrac{1}{4} \kappa^{CD} \nabla_{DG'}\phi_{ABFC} + \tfrac{3}{4} \phi_{CD(AB} \nabla^{D}{}_{\vert G'\vert}\kappa_{F)}{}^{C} \\
&=- \tfrac{1}{4} \phi_{ABFD} \nabla_{CG'}\kappa^{CD}  -  \tfrac{1}{4} \kappa^{CD} \nabla_{DG'}\phi_{ABFC}  -  \tfrac{1}{4} \kappa_{A}{}^{C} \nabla_{DG'}\phi_{BFC}{}^{D} + \tfrac{1}{4} \phi_{BFCD} \nabla^{D}{}_{G'}\kappa_{A}{}^{C} + \tfrac{1}{4} \phi_{AFCD} \nabla^{D}{}_{G'}\kappa_{B}{}^{C} + \tfrac{1}{4} \phi_{ABCD} \nabla^{D}{}_{G'}\kappa_{F}{}^{C}
\end{align*}
the second equality following from $\nabla^A{}_{A'}\phi_{ABCD}=0$. Using \eqref{Eq:DecompGradKS}, the definition of $F_{A'ABC}$ and rearranging, 
\begin{equation}
 \kappa^{CD} \nabla_{DG'}\phi_{ABFC} = 2 \xi^{C}{}_{G'} \phi_{ABFC} +
 \phi_{CD(AB} H_{\vert G'\vert F)}{}^{CD} - 4
 F_{G'ABF}\label{Eq:MiscIdentity}
 \end{equation}
Then, using the irreducible decomposition
\begin{equation*}
\kappa_{F}{}^{D} \phi_{ABCD} = B_{ABCF} + \tfrac{1}{4} \kappa^{DG} \phi_{BCDG} \epsilon_{AF} + \tfrac{1}{4} \kappa^{DG} \phi_{ACDG} \epsilon_{BF} + \tfrac{1}{4} \kappa^{DG} \phi_{ABDG} \epsilon_{CF},	
\end{equation*}
one calculates
\begin{align*}
	\kappa_{A}{}^{F} \nabla_{GF'}\phi_{BCDF} &= \nabla_{GF'}(\kappa_{A}{}^{F}\phi_{BCDF}) - \phi_{BCDF}\nabla_{GF'}\kappa_{A}{}^{F}\\
	&= \nabla_{GF'}B_{ABCD} -     \tfrac{3}{4} \kappa^{FH} \epsilon_{A(B} \nabla_{\vert GF'\vert}\phi_{CD)FH}\\
	&\qquad -  \phi_{BCDF} \nabla_{GF'}\kappa_{A}{}^{F} -  \tfrac{3}{4} \epsilon_{A(B}\phi_{CD)FH} \nabla_{GF'}\kappa^{FH}\\
	&=\nabla_{GF'}B_{ABCD} + \tfrac{1}{3} \xi_{AF'} \phi_{BCDG} -  \tfrac{1}{3} \phi_{BCDF} H_{F'AG}{}^{F} + \tfrac{1}{2} \xi^{F}{}_{F'}\epsilon_{A(B} \phi_{CD)GF}  \\
	&\qquad -  \tfrac{1}{4} \epsilon_{A(B}\phi_{CD)FH} H_{F'G}{}^{FH}   -  \tfrac{1}{3} \xi^{F}{}_{F'} \phi_{BCDF} \epsilon_{AG}  -  \tfrac{3}{4} \kappa^{FH} \epsilon_{A(B} \nabla_{\vert GF'\vert}\phi_{CD)FH}
\end{align*}
where the third equality follows from \eqref{Eq:DecompGradKS}. Then swapping indices on the $\nabla\bm\phi$ terms and using $\nabla^A{}_{A'}\phi_{ABCD}=0$, one obtains \eqref{Eq:MiscIdentity2}:
\begin{align*}
    \kappa_{A}{}^{F} \nabla_{FF'}\phi_{BCDG} &= \tfrac{1}{3} \xi_{AF'}
    \phi_{BCDG} - \tfrac{1}{3} \xi^{F}{}_{F'} \phi_{BCGF}
    \epsilon_{AD} - \tfrac{1}{3}
    \xi^{F}{}_{F'}\phi_{(BC|DF}\epsilon_{A|G)} \nonumber \\ & -
    \tfrac{2}{3} \xi^{F}{}_{F'}\phi_{(B|D|C|F}\epsilon_{A|G)} +
    \epsilon_{AD} F_{F'BCG} + \nabla_{(B|F'}B_{A|CG)D} + \tfrac{1}{3}
    \phi_{(BC|D}{}^{F}H_{F'A|G)F} \nonumber \\& + 2
    \epsilon_{A(B}F_{|F'\vert CG)D} - \tfrac{1}{3}
    \phi_{(BC}{}^{FH}H_{|F'|G)FH}\epsilon_{AD} - \tfrac{1}{6}
    \phi_{(BC}{}^{FH}H_{|F'DFH}\epsilon_{A|G)} \nonumber\\ & -
    \tfrac{1}{3} \phi_{(B|D}{}^{FH}H_{F'|C|FH}\epsilon_{A|G)} -
    \tfrac{1}{6}
    \phi_{D(B}{}^{FH}H_{|F'|C|FH}\epsilon_{A|G)}.
\end{align*}
To show \eqref{Eq:MiscIdentity3}, we start with ...Using the irreducible decomposition
\begin{equation*}
\phi_{ABCD} \phi_{FG}{}^{CD} = \tfrac{1}{6} \phi_{CDHL} \phi^{CDHL}
\epsilon_{AG} \epsilon_{BF} + \tfrac{1}{6} \phi_{CDHL} \phi^{CDHL}
\epsilon_{AF} \epsilon_{BG} +
\phi_{(AB}{}^{CD}\phi_{FG)CD} \label{Eq:DecompPhiSquared}
\end{equation*}
we can show that
\begin{align*}
    \kappa^{AD} \phi_{AD}{}^{GH} \nabla_{HA'}\phi_{BCFG} &= 4
    \xi^{A}{}_{A'}\phi_{(BC}{}^{DG}\phi_{F)ADG} + \tfrac{1}{2}
    \phi_{ADGH} \phi^{ADGH} H_{A'BCF} \nonumber\\ & - 4
    B_{(B}{}^{ADG}\nabla_{|AA'|}\phi_{CF)DG} - 8
    \phi_{(BC}{}^{AD}F_{|A'|F)ADG}\nonumber \\ & - 4
    \phi_{(B}{}^{ADG}\nabla_{|AA'|}B_{CF)DG} - \tfrac{1}{3}
    \phi_{(BC}{}^{AD}\phi_{|AD}{}^{GH}H_{A'|F)GH}\nonumber \\ & -
    \tfrac{2}{3}
    \phi_{(B}{}^{ADG}\phi_{C|AD}{}^{H}H_{A'|F)GH} \label{Eq:MiscIdentity3}
\end{align*}

%\subsection{Spin dyad calculations}



\subsection{Proof of Proposition \ref{prop_remove_redundant_conditions}}
\label{Sec:ProofOfProp3}

%In this section the proof of Proposition
%\ref{prop_remove_redundant_conditions} is given.
The Buchdahl constraint restricts the Petrov type of
$\bm\phi$ to be type $D$, $N$ or $O$.
The calculation proceeds by expanding out the
conditions \eqref{Raw_CSKID} in a spin dyad
by considering separately the cases $\kappa_{AB}\kappa^{AB}\equiv 0$
and $\kappa_{AB}\kappa^{AB}\neq 0$ on $\mathcal{U}$, corresponding to Petrov types N and D, respectively.
Observe that for Type $O$, for which $\phi_{ABCD}=0$,
the proof of \ref{prop_remove_redundant_conditions} trivialises,
so only the type $N$ and $D$ are needed.
\\

Recalling that $\mathcal{D}_{AB}:= \tau_{(A}{}^{A'}\nabla_{B)A'}$, a straightforward computation yields 
\begin{align*}
& o^Ao^Bo^C\mathcal{D}_{AB}o_C = -\sigma,\\
& o^Ao^B\iota^C\mathcal{D}_{AB}o_C=o^Ao^Bo^C\mathcal{D}_{AB}\iota_C=-\beta,\\
& o^A\iota^B o^C\mathcal{D}_{AB}o_C = \tfrac{1}{2}(\kappa - \tau),\\
& o^A\iota^B\iota^C\mathcal{D}_{AB}o_C = o^A\iota^B o^C\mathcal{D}_{AB}\iota_C = \tfrac{1}{2}(\epsilon - \gamma),\\
& \iota^A\iota^B o^C\mathcal{D}_{AB}o_C = \rho,\\
& \iota^A\iota^B\iota^C\mathcal{D}_{AB}o_C = \iota^A\iota^B o^C\mathcal{D}_{AB}\iota_C = \alpha,\\
& o^A o^B\iota^C\mathcal{D}_{AB}\iota_C = \mu,\\
& o^A\iota^B\iota^C\mathcal{D}_{AB} = \tfrac{1}{2}(\pi - \nu),\\
& \iota^A\iota^B\iota^C\mathcal{D}_{AB}\iota_C = \lambda	,
\end{align*}
%\[o^Ao^Bo^C\mathcal{D}_{AB}o_C = -\sigma, \qquad o^A\iota^B o^C\mathcal{D}_{AB}o_C = \tfrac{1}{2}(\kappa - \tau), \qquad \iota^A\iota^B o^C\mathcal{D}_{AB}o_C = \rho,\]
%\[o^A o^B\iota^C\mathcal{D}_{AB}\iota_C = \mu, \qquad o^A\iota^B\iota^C\mathcal{D}_{AB} = \tfrac{1}{2}(\pi - \nu), \qquad \iota^A\iota^B\iota^C\mathcal{D}_{AB}\iota_C = \lambda	\]
%\[o^Ao^B\iota^C\mathcal{D}_{AB}o_C=o^Ao^Bo^C\mathcal{D}_{AB}\iota_C=-\beta, \qquad o^A\iota^B\iota^C\mathcal{D}_{AB}o_C = o^A\iota^B o^C\mathcal{D}_{AB}\iota_C = \tfrac{1}{2}(\epsilon - \gamma), \]
%\[\iota^A\iota^B\iota^C\mathcal{D}_{AB}o_C = \iota^A\iota^B o^C\mathcal{D}_{AB}\iota_C = \alpha, \]
where we are following the conventions of [] in the definition of the Newmann--Penrose (NP) scalars $\alpha, \beta, \epsilon, \gamma, \kappa, \mu, 
\lambda, \rho, \tau, \sigma, \nu, \pi.$
\\

The following identities will be useful
\begin{subequations}
\begin{eqnarray}
&& \mathcal{D}_{AB}o^B = o^Bo^C\mathcal{D}_{AC}\iota_B - \iota^B o^C\mathcal{D}_{AB}o_C,\\
&& \mathcal{D}_{AB}\iota^B = \iota^B o^C(\mathcal{D}_{AC}\iota_B - \mathcal{D}_{AB}\iota_C),
\end{eqnarray}
\end{subequations}
and follow easily from $\epsilon_{AB}=o_A\iota_B - o_B\iota_A$. 

\subsubsection{Case I: $\kappa_{AB}\kappa^{AB}\equiv 0$ on $\mathcal{U}$} \label{TypeNCase}

\mnotex{Recovers twistor case, right?}
The assumption that $\kappa_{AB}\kappa^{AB}=0$ on $\mathcal{U}$ implies that there exists a spin dyad $\lbrace \bm\omicron, \bm\iota\rbrace$ on $\mathcal{U}$ such that $\kappa_{AB}=\omicron_A\omicron_B$. The Buchdahl constraint then implies that 
\begin{equation}
\phi_{ABCD}=\phi \omicron_A\omicron_B\omicron_C\omicron_D
\end{equation}
for some scalar field $\phi:\mathcal{U}\rightarrow\mathbb{C}$ and hence that the curvature is of Petrov type N on $\mathcal{U}$. Note that $\phi_{ABCD}\omicron^D=0$. The equation $\mathcal{D}_{(AB}\kappa_{CD)}=0$ implies 
\begin{equation*}
 \omicron^A \omicron^B \omicron^C \mathcal{D}_{AB}\omicron_C= \omicron^A \omicron^B \iota^C\mathcal{D}_{(AB}\omicron_{C)}= \omicron^A \iota^B \iota^C\mathcal{D}_{(AB}\omicron_{C)}= \iota^A \iota^B \iota^C\mathcal{D}_{AB}\omicron_{C}=0,
\end{equation*}
implying that $\mathcal{D}_{(AB}\omicron_{C)}=0$ ---that is to say that $\omicron_A$ is a twistor candidate. In terms of the NP scalars, the above read as follows 
\begin{equation}\label{NPRelationsTypeN} 
\sigma=- \beta + \kappa -  \tau=\epsilon -  \gamma + \rho=\alpha=0.
\end{equation}	
Using these relations, we obtain
\begin{equation}\label{AuxSpinorsTypeN}
 \xi = -3\beta,\qquad \xi_{AB} &= 2\rho o_Ao_B - 2\beta o_{(A}\iota_{B)}.
\end{equation}
The non-trivial component of the constraint $\mathcal{D}^{AB}\phi_{ABCD}=0$ reduces to 
\begin{equation}
 \mathcal{D}_{\bm0\bm0}\phi = \tfrac{5}{3}\phi (2\beta  +  \kappa -  \tau)=5\phi\beta.\label{TypeNweylConstraint}
\end{equation}
Then, substituting \eqref{AuxSpinorsTypeN}, condition \eqref{RedundantCondition1} reduces to
\begin{align*}
\omicron^A \iota^B \iota^C \iota^D \left(2\kappa_{(A}{}^{F}\mathcal{D}_{B}{}^{G}\phi_{CD)FG} + \phi_{(ABC}{}^{F}\xi_{D)F}\right) &=\tfrac{1}{2}\phi\sigma =0,\\
\iota^A\iota^B \iota^C \iota^D \left(2\kappa_{(A}{}^{F}\mathcal{D}_{B}{}^{G}\phi_{CD)FG} + \phi_{(ABC}{}^{F}\xi_{D)F}\right) &=\phi(\beta -  \kappa  + \tau) = 0
\end{align*}
with the second equality following from \eqref{NPRelationsTypeN} and
with all other components vanishing trivially. These are essentially the same computations as in []. On the other hand, substituting \eqref{AuxSpinorsTypeN}, condition \eqref{RedundantCondition2} reduces to
\begin{align*}
\omicron^A\omicron^B\iota^C\iota^D\left(\xi^{FG}\mathcal{D}_{FG}\phi_{ABCD} + \tfrac{2}{3}\xi\mathcal{D}_{(A}{}^F\phi_{BCD)F}\right)&=\tfrac{1}{3}\phi \xi \sigma=0,\\
\omicron^A\iota^B\iota^C\iota^D\left(\xi^{FG}\mathcal{D}_{FG}\phi_{ABCD} + \tfrac{2}{3}\xi\mathcal{D}_{(A}{}^F\phi_{BCD)F}\right)&= \tfrac{1}{2} \beta \mathcal{D}_{\bm0\bm0}\phi-2 \beta^2 \phi -  \tfrac{1}{2} \beta \kappa \phi - 2 \rho \sigma \phi + \tfrac{1}{2} \beta \tau \phi =0,\\
\iota^A\iota^B\iota^C\iota^D\left(\xi^{FG}\mathcal{D}_{FG}\phi_{ABCD} + \tfrac{2}{3}\xi\mathcal{D}_{(A}{}^F\phi_{BCD)F}\right) &= 2 \rho \mathcal{D}_{\bm0\bm0}\phi-10 \beta \rho \phi =0,
\end{align*}
where we are again using \eqref{NPRelationsTypeN} and \eqref{TypeNweylConstraint}. All other components vanishing trivially. Hence, in this case, both conditions \eqref{RedundantCondition1} and \eqref{RedundantCondition2} trivialise. 


\subsubsection{Case II: $\kappa_{AB}\kappa^{AB}\neq 0$ on $\mathcal{U}$}\label{TypeDCase}

There exists a spin dyad $\lbrace \bm\omicron, \bm\iota\rbrace$ such that $\kappa_{AB} = e^{\varkappa} \omicron_{(A}\iota_{B)}$ for some $\varkappa:\mathcal{U}\rightarrow\mathbb{C}$. The Buchdahl constraint then implies that the rescaled Weyl spinor takes the form
\[ \phi_{ABCD}= \phi \omicron_{(A}\omicron_B\iota_C\iota_{D)}\]
for some scalar field $\phi:\mathcal{U}\rightarrow\mathbb{C}$ and hence that the curvature is of Petrov type D on $\mathcal{U}$. The equation $\mathcal{D}_{(AB}\kappa_{CD)}=0$ is equivalent to 
\begin{subequations}
\begin{eqnarray}
&& \sigma=0,\label{SpatialSenTypeD0000}\\
&& \mathcal{D}_{\bm0\bm0}\varkappa = \tau - \kappa, \label{SpatialSenTypeD0001}\\
&& \mathcal{D}_{\bm0\bm1}\varkappa = \tfrac{1}{2} (\mu-\rho),\label{SpatialSenTypeD0011}\\
&& \mathcal{D}_{\bm1\bm1}\varkappa = \pi - \nu, \label{SpatialSenTypeD0111}\\
&& \lambda = 0.\label{SpatialSenTypeD1111}
\end{eqnarray}
\end{subequations}
Using the above, the auxiliary spinors can be written as 
\begin{subequations}
\begin{eqnarray} 
&& \xi =\tfrac{3}{2}e^{\varkappa}(\mu -  \rho), \label{AuxSpinor1TypeD}\\
&& \xi_{AB} = e^{\varkappa} (\nu - \pi) o_{A} o_{B} + e^{\varkappa}(\mu + \rho) o_{(A}\iota_{B)} + e^{\varkappa} (\tau - \kappa) \iota_{A} \iota_{B}. \label{AuxSpinor2TypeD}
\end{eqnarray}
\end{subequations}
The constraint $\mathcal{D}^{CD}\phi_{CDAB}=0$ is equivalent to 
\begin{equation}\label{WeylConstraintTypeD}
  \mathcal{D}_{\bm0\bm0}\phi = 3 \phi(\kappa - \tau), \qquad 
  \mathcal{D}_{\bm0\bm1}\phi = \tfrac{3}{2}\phi(\rho -\mu), \qquad
  \mathcal{D}_{\bm1\bm1}\phi = 3\phi (\nu  - \pi). 
\end{equation}
Then, substituting \eqref{AuxSpinor1TypeD} and \eqref{AuxSpinor2TypeD}, condition \eqref{RedundantCondition1} decomposes as follows
\begin{align*}
\omicron^A \omicron^B \omicron^C \omicron^D \left(2\kappa_{(A}{}^{F}\mathcal{D}_{B}{}^{G}\phi_{CD)FG} + \phi_{(ABC}{}^{F}\xi_{D)F}\right) 
&=- \tfrac{1}{2} e^{\varkappa}\phi \sigma=0,\\
\omicron^A \omicron^B \omicron^C \iota^D \left(2\kappa_{(A}{}^{F}\mathcal{D}_{B}{}^{G}\phi_{CD)FG} + \phi_{(ABC}{}^{F}\xi_{D)F}\right) & =\tfrac{1}{24} e^{\varkappa} \mathcal{D}_{\bm0\bm0}\phi + \tfrac{1}{8} e^{\varkappa}\phi (\tau - \kappa)  =0,\\
 \omicron^A \omicron^B \iota^C \iota^D \left(2\kappa_{(A}{}^{F}\mathcal{D}_{B}{}^{G}\phi_{CD)FG} + \phi_{(ABC}{}^{F}\xi_{D)F}\right) 
& = \tfrac{1}{18} e^{\varkappa} \mathcal{D}_{\bm0\bm1}\phi + \tfrac{1}{12} e^{\varkappa}\phi (\mu - \rho)  = 0,\\
 \omicron^A \iota^B \iota^C \iota^D \left(2\kappa_{(A}{}^{F}\mathcal{D}_{B}{}^{G}\phi_{CD)FG} + \phi_{(ABC}{}^{F}\xi_{D)F}\right) & =\tfrac{1}{24} e^{\varkappa} \mathcal{D}_{\bm1\bm1}\phi + \tfrac{1}{8} e^{\varkappa} \phi(\pi-\nu)  = 0,\\
 \iota^A \iota^B \iota^C \iota^D \left(2\kappa_{(A}{}^{F}\mathcal{D}_{B}{}^{G}\phi_{CD)FG} + \phi_{(ABC}{}^{F}\xi_{D)F}\right) & = - \tfrac{1}{2} e^{\kappa} \phi \lambda = 0,
\end{align*}
the equality with zero following from \eqref{WeylConstraintTypeD}. 
On the other hand, substituting \eqref{AuxSpinor1TypeD} and \eqref{AuxSpinor2TypeD} into \eqref{RedundantCondition2}, 
\begin{align*}
& o^Ao^Bo^Co^D(\xi^{FG}\mathcal{D}_{FG}\phi_{ABCD} + \tfrac{2}{3} \xi \mathcal{D}_{(A}{}^{F}\phi_{BCD)F})  =\tfrac{1}{3} \phi \sigma  \xi =0,\\
& o^Ao^Bo^C\iota^D(\xi^{FG}\mathcal{D}_{FG}\phi_{ABCD} + \tfrac{2}{3} \xi \mathcal{D}_{(A}{}^{F}\phi_{BCD)F}) \\
&\qquad\qquad\qquad\qquad\qquad = \tfrac{1}{8} e^{\varkappa} (\rho-\mu) \mathcal{D}_{\bm0\bm0}\phi + \tfrac{3}{8}e^{\varkappa}\phi  (\kappa-\tau)(\mu - \rho)  +  \tfrac{1}{2} e^{\varkappa}\phi(\pi-\nu) \sigma   = 0	,\\
& o^Ao^B\iota^C\iota^D(\xi^{FG}\mathcal{D}_{FG}\phi_{ABCD} + \tfrac{2}{3} \xi \mathcal{D}_{(A}{}^{F}\phi_{BCD)F})  \\
&\qquad\qquad\qquad\qquad\qquad=	\tfrac{1}{6} e^{\varkappa}\left((\nu-\pi)\mathcal{D}_{\bm0\bm0}\phi + (\rho+\mu)\mathcal{D}_{\bm0\bm1}\phi + (\tau-\kappa)\mathcal{D}_{\bm1\bm1}\phi \right)  + \tfrac{1}{4} e^{\varkappa} \phi(\mu^2-\rho^2) = 0,\\
& o^A\iota^B\iota^C\iota^D(\xi^{FG}\mathcal{D}_{FG}\phi_{ABCD} + \tfrac{2}{3} \xi \mathcal{D}_{(A}{}^{F}\phi_{BCD)F})  \\
& \qquad\qquad\qquad\qquad\qquad =	\tfrac{1}{8} e^{\varkappa}  (\mu - \rho) \mathcal{D}_{\bm1\bm1}\phi  +  \tfrac{3}{8} e^{\varkappa} \phi(\mu-\rho) (\pi-\nu) + \tfrac{1}{2} e^{\varkappa}\phi (\kappa-\tau) \lambda    = 0,\\
& \iota^A\iota^B\iota^C\iota^D(\xi^{FG}\mathcal{D}_{FG}\phi_{ABCD} + \tfrac{2}{3} \xi \mathcal{D}_{(A}{}^{F}\phi_{BCD)F})  =	- \tfrac{1}{3}\phi \lambda \xi = 0.
\end{align*}
Hence, again, \eqref{RedundantCondition1} and \eqref{RedundantCondition2} trivialise. Combining the result of this section with the previous, Proposition \ref{prop_remove_redundant_conditions} follows immediately.


%%%%%%%%%%%%%% 
%% \bibliographystyle{/home/egarcia/Dropbox/References/reporthack}
%% \bibliography{/home/egarcia/Dropbox/References/GRbibJune2021a}
%%%%%%%%%%%%%%


\begin{thebibliography}{10}

\bibitem{BaeVal10a}
T.~B\"{a}ckdahl \& J.~A. {Valiente Kroon},
\newblock {\em Geometric invariant measuring the deviation from Kerr data},
\newblock Phys. Rev. Lett. {\bf 104}, 231102 (2010).

\bibitem{BaeVal10b}
T.~B\"{a}ckdahl \& J.~A. {Valiente Kroon},
\newblock {\em On the construction of a geometric invariant measuring the
  deviation from Kerr data},
\newblock Ann. Henri Poincar\'e {\bf 11}, 1225 (2010).

\bibitem{BaeVal11b}
T.~B\"{a}ckdahl \& J.~A. {Valiente Kroon},
\newblock {\em The "non-Kerrness" of domains of outer communication of black
  holes and exteriors of stars},
\newblock Proc. Roy. Soc. Lond. A {\bf 467}, 1701 (2011).

\bibitem{BeiChr97b}
R.~Beig \& P.~T. Chru\'{s}ciel,
\newblock {\em Killing initial data},
\newblock Class. Quantum Grav. {\bf 14}, A83 (1997).

\bibitem{CarVal18}
D.~A. Carranza \& J.~A. Valiente~Kroon,
\newblock {\em Killing boundary data for anti-de Sitter-like spacetimes},
\newblock Classical and Quantum Gravity {\bf 35}(15), 155011 (Jul 2018).

\bibitem{ChrPaetz13}
P.~T.~Chru\'{s}ciel \& T.-T.~Paetz, 
\newblock {\em KIDs like cones},
\newblock Classical and Quantum Gravity 30.23 (2013): 235036.

\bibitem{ColRacVal18}
M.~J. Cole, I.~Rácz, \& J.~A. Valiente~Kroon,
\newblock {\em Killing spinor data on distorted black hole horizons and the
  uniqueness of stationary vacuum black holes},
\newblock Classical and Quantum Gravity {\bf 35}(20), 205001 (Sep 2018).

\bibitem{ValCol16}
M.~J. Cole \& J.~A.~V. Kroon,
\newblock {\em Killing spinors as a characterisation of rotating black hole
  spacetimes},
\newblock  {\bf 33}(12), 125019 (may 2016).

\bibitem{Fri81b}
H.~Friedrich,
\newblock {\em The asymptotic characteristic initial value problem for
  {Einstein}'s vacuum field equations as an initial value problem for a
  first-order quasilinear symmetric hyperbolic system},
\newblock Proc. Roy. Soc. Lond. A {\bf 378}, 401 (1981).

\bibitem{Fri81a}
H.~Friedrich,
\newblock {\em On the regular and the asymptotic characteristic initial value
  problem for {Einstein}'s vacuum field equations},
\newblock Proc. Roy. Soc. Lond. A {\bf 375}, 169 (1981).

\bibitem{Fri82}
H.~Friedrich,
\newblock {\em On the existence of analytic null asymptotically flat solutions
  of {Einstein}'s vacuum field equations},
\newblock Proc. Roy. Soc. Lond. A {\bf 381}, 361 (1982).

\bibitem{Fri83}
H.~Friedrich,
\newblock {\em Cauchy problems for the conformal vacuum field equations in
  General Relativity},
\newblock Comm. Math. Phys. {\bf 91}, 445 (1983).

\bibitem{Fri86c}
H.~Friedrich,
\newblock {\em {Existence and structure of past asymptotically simple solutions
  of Einstein's field equations with positive cosmological constant}},
\newblock J. Geom. Phys. {\bf 3}, 101 (1986).

\bibitem{Fri86b}
H.~Friedrich,
\newblock {\em {On the existence of n-geodesically complete or future complete
  solutions of Einstein's field equations with smooth asymptotic structure}},
\newblock Comm. Math. Phys. {\bf 107}, 587 (1986).

\bibitem{GarKha19}
A.~García-Parrado \& I.~Khavkine,
\newblock {\em Conformal Killing initial data},
\newblock Journal of Mathematical Physics {\bf 60}(12), 122502 (Dec 2019).

\bibitem{GasVal15}
E.~{Gasperin} \& J.~A. {Valiente Kroon},
\newblock {\em {Spinorial wave equations and stability of the Milne
  spacetime}},
\newblock Classical and Quantum Gravity  (2015).

\bibitem{GasVal17}
E.~Gasperin \& J.~A. Valiente~Kroon,
\newblock {\em Perturbations of the Asymptotic Region of the Schwarzschild--de
  Sitter Spacetime},
\newblock Annales Henri Poincar{\'e} , 1--73 (2017).

\bibitem{GasVal17a}
E.~Gasperin \& J.~A.~V. Valiente~Kroon,
\newblock {\em {Polyhomogeneous expansions from time symmetric initial data}},
\newblock Class. Quant. Grav. {\bf 34}(19), 195007 (2017).

\bibitem{GarVal08c}
A.~G.-P. Gómez-Lobo \& J.~A. {Valiente Kroon},
\newblock {\em Killing spinor initial data sets},
\newblock Journal of Geometry and Physics {\bf 58}(9), 1186--1202 (2008).

\bibitem{KatLevDav69}
G.~H. Katzin, J.~Levine, \& W.~R. Davis,
\newblock {\em Curvature Collineations: A Fundamental Symmetry Property of the
  Space‐Times of General Relativity Defined by the Vanishing Lie Derivative
  of the Riemann Curvature Tensor},
\newblock Journal of Mathematical Physics {\bf 10}(4), 617--629 (1969).

\bibitem{LueVal09}
C.~L\"ubbe \& J.~A. {Valiente Kroon},
\newblock {\em On de {S}itter-like and {M}inkowski-like spacetimes},
\newblock Class. Quantum Grav. {\bf 26}, 145012 (2009).

\bibitem{Mar99}
M.~Mars,
\newblock {\em A spacetime characterization of the Kerr metric},
\newblock Class. Quantum Grav. {\bf 16}, 2507 (1999).

\bibitem{Mar00}
M.~Mars,
\newblock {\em Uniqueness properties of the Kerr metric},
\newblock Class. Quantum Grav. {\bf 17}, 3353 (2000).

\bibitem{MarPaeSenSim16}
M.~{Mars}, T.-T. {Paetz}, J.~M.~M. {Senovilla}, \& W.~{Simon},
\newblock {\em {Characterization of (asymptotically) Kerr-de Sitter-like
  spacetimes at null infinity}},
\newblock Classical and Quantum Gravity {\bf 33}(15), 155001 (Aug. 2016).

\bibitem{McLBer93}
R.~G. McLenaghan \& N.~V. den Bergh,
\newblock {\em Spacetimes admitting Killing 2-spinors},
\newblock Classical and Quantum Gravity {\bf 10}, 2179--2185 (1993).

\bibitem{Pae13}
T.-T. Paetz,
\newblock {\em Conformally covariant systems of wave equations and their
  equivalence to {E}instein's field equations},
\newblock Ann. Henri Poincar\'{e} {\bf 16}, 2059 (2013).

\bibitem{Pae14a}
T.-T. {Paetz},
\newblock {\em {KIDs prefer special cones}},
\newblock Classical and Quantum Gravity {\bf 31}(8), 085007 (Apr. 2014).

\bibitem{PenRin84}
R.~Penrose \& W.~Rindler,
\newblock {\em Spinors and space-time. {V}olume 1. {T}wo-spinor calculus and
  relativistic fields},
\newblock Cambridge University Press, 1984.

\bibitem{PenRin86}
R.~Penrose \& W.~Rindler,
\newblock {\em Spinors and space-time. {V}olume 2. {S}pinor and twistor methods
  in space-time geometry},
\newblock Cambridge University Press, 1986.

\bibitem{Rob75b}
D.~C. Robinson,
\newblock {\em Uniqueness of the Kerr black hole},
\newblock Phys. Rev. Lett. {\bf 34}, 905 (1975).

\bibitem{Sim84}
W.~Simon,
\newblock {\em Characterizations of the Kerr metric},
\newblock Gen. Rel. Grav. {\bf 16}, 465 (1984).

\bibitem{Som80}
P.~Sommers,
\newblock {\em Space spinors},
\newblock J. Math. Phys. {\bf 21}, 2567 (1980).

\bibitem{Ste91}
J.~Stewart,
\newblock {\em Advanced general relativity},
\newblock Cambridge University Press, 1991.

\bibitem{Tay96c}
M.~E. Taylor,
\newblock {\em Partial differential equations {III}: nonlinear equations},
\newblock Springer Verlag, 1996.

\bibitem{CFEbook}
J.~A. {Valiente Kroon},
\newblock {\em Conformal methods in General Relativity},
\newblock Cambridge University Press, 2016.

\bibitem{WalkerPenrose70}
M.~Walker \& R.~Penrose,
\newblock {\em On quadratic first integrals of the geodesic equations for type {22} spacetimes.}
\newblock Communications in Mathematical Physics 18.4 (1970): 265-274.

\bibitem{Wei90a}
G.~Weinstein,
\newblock {\em On rotating black holes in equilibrium in general relativity},
\newblock Communications on Pure and Applied Mathematics {\bf 43}(7), 903--948
  (1990).

\end{thebibliography}




\end{document}
