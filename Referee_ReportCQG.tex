

%%%% answer to referee report 14 June 2022

\documentclass[a4paper,10pt]{article}      
\usepackage[utf8x]{inputenc}
\pdfoutput=1
\usepackage{nccfoots}
\renewcommand\refname{Referencias}
\usepackage{times}
\usepackage{geometry}
\usepackage[T1]{fontenc}
\usepackage{color}
\usepackage{amsmath}
\usepackage{amssymb}
\usepackage{array}
\usepackage{amsthm}
\usepackage{graphicx}
\usepackage{hyperref}
\usepackage{setspace}
\usepackage{stmaryrd}
\usepackage{marginnote}
\usepackage[mathscr]{euscript}
\usepackage{subfigure}
\usepackage{graphicx}
\usepackage{latexsym}
\usepackage[dvips]{epsfig}
%\usepackage{showkeys}
\usepackage{wasysym}
\usepackage{mathrsfs}
\usepackage{eufrak}
\usepackage{bm}
\usepackage{authblk}
\usepackage{slashed}
\usepackage{yhmath} 
%\usepackage{authblk}
%%%%%large commented parts
\usepackage{verbatim}
\renewcommand\Affilfont{\itshape\small}

\usepackage{natbib}


\theoremstyle{plain}
\newtheorem{proposition}{Proposition}
\newtheorem{lemma}{Lemma}
\newtheorem{theorem}{Theorem}
\newtheorem{assumption}{Assumption}
\newtheorem*{conjecture}{Conjecture}
\newtheorem*{subconjecture}{Subconjecture}
\newtheorem{corollary}{Corollary}
\newtheorem{main}{Main Result}
\newtheorem*{definition}{Definition}
\newtheorem*{remark}{Remark}



\usepackage{tikz}
\usetikzlibrary{arrows}

\newcommand{\myrule} [3] []{
    \begin{center}
        \begin{tikzpicture}
            \draw[#2-#3, ultra thick, #1] (0,0) to (0.5\linewidth,0);
        \end{tikzpicture}
    \end{center}
}


% Underlined lowcase latin letters
\def\es{{\bar{s}}}
\def\er{{\bar{r}}}

% Boldface mathmode lowcase latin letters
\def\bma{{\bm a}}
\def\bmb{{\bm b}}
\def\bmc{{\bm c}}
\def\bmd{{\bm d}}
\def\bme{{\bm e}}
\def\bmf{{\bm f}}
\def\bmg{{\bm g}}
\def\bmh{{\bm h}}
\def\bmi{{\bm i}}
\def\bmj{{\bm j}}
\def\bmk{{\bm k}}
\def\bml{{\bm l}}
\def\bmn{{\bm n}}
\def\bmm{{\bm m}}
\def\bmo{{\bm o}}
\def\bmp{{\bm p}}
\def\bms{{\bm s}}
\def\bmt{{\bm t}}
\def\bmu{{\bm u}}
\def\bmv{{\bm v}}
\def\bmw{{\bm w}}
\def\bmx{{\bm x}}
\def\bmy{{\bm y}}
\def\bmz{{\bm z}}

% Boldface mathmode numbers
\def\bmzero{{\bm 0}}
\def\bmone{{\bm 1}}
\def\bmtwo{{\bm 2}}
\def\bmthree{{\bm 3}}

% Boldface mathmode uppercase latin letters
\def\bmA{{\bm A}}
\def\bmB{{\bm B}}
\def\bmC{{\bm C}}
\def\bmD{{\bm D}}
\def\bmE{{\bm E}}
\def\bmF{{\bm F}}
\def\bmG{{\bm G}}
\def\bmH{{\bm H}}
\def\bmK{{\bm K}}
\def\bmL{{\bm L}}
\def\bmM{{\bm M}}
\def\bmN{{\bm N}}
\def\bmP{{\bm P}}
\def\bmQ{{\bm Q}}
\def\bmR{{\bm R}}
\def\bmS{{\bm S}}
\def\bmT{{\bm T}}
\def\bmX{{\bm X}}
\def\bmZ{{\bm Z}}

\def\Riem{{\bm R}{\bm i}{\bm e}{\bm m}}
\def\Ric{{\bm R}{\bm i}{\bm c}}
\def\Weyl{{\bm W}{\bm e}{\bm y}{\bm l}}
\def\Schouten{{\bm S}{\bmc}{\bm h}}

% Mathbf letters
\def\mbfu{\mathbf{u}}

% Boldface mathmode lowcase greek letters
\def\bmalpha{{\bm \alpha}}
\def\bmbeta{{\bm \beta}}
\def\bmgamma{{\bm \gamma}}
\def\bmdelta{{\bm \delta}}
\def\bmepsilon{{\bm \epsilon}}
\def\bmeta{{\bm \eta}}
\def\bmzeta{{\bm\zeta}}
\def\bmxi{{\bm \xi}}
\def\bmchi{{\bm \chi}}
\def\bmiota{{\bm \iota}}
\def\bmomega{{\bm \omega}}
\def\bmlambda{{\bm \lambda}}
\def\bmmu{{\bm \mu}}
\def\bmnu{{\bm \nu}}
\def\bmpi{{\bm \pi}}
\def\bmpsi{{\bm \psi}}
\def\bmphi{{\bm \phi}}
\def\bmvarphi{{\bm \varphi}}
\def\bmsigma{{\bm \sigma}}
\def\bmvarsigma{{\bm \varsigma}}
\def\bmtau{{\bm \tau}}

% Boldface mathmode uppercase greek letters
\def\bmUpsilon{{\bm \Upsilon}}
\def\bmSigma{{\bm \Sigma}}

% Boldface operators
\def\bmpartial{{\bm \partial}}
\def\bmnabla{{\bm \nabla}}
\def\bmhbar{{\bm \hbar}}
\def\bmperp{{\bm \perp}}


%Yang-Mills indices
\def\fraka{\mathfrak{a}}
\def\frakb{\mathfrak{b}}
\def\frakc{\mathfrak{c}}


%Counter variable for the margin notes
\newcounter{mnotecount}%[section]

% This code generates the margin notes
\newcommand{\mnotex}[1]%{}
{\protect{\stepcounter{mnotecount}}$^{\mbox{\footnotesize $\bullet$\themnotecount}}$ 
\marginpar{%\color{red}%
\raggedright\tiny\em
$\!\!\!\!\!\!\,\bullet$\themnotecount: #1} }


\renewcommand{\title}[1]{{\bfseries #1}\par}
\renewcommand{\author}[1]{\medskip{#1}\par\smallskip}
\newcommand{\affiliation}[1]{{\itshape #1}\par}
\newcommand{\email}[1]{E-mail:~\texttt{#1}\par}


\numberwithin{equation}{section}




\begin{document}
\begin{center}
  \title{\LARGE Response to the referee report on ``The conformal Killing spinor initial data equations''
  \\ \vspace{0.5cm} \Large
  }
\vspace{3mm}
\author{\large }
%\Footnotetext{}{1991 \emph{Mathematics Subject Classification.} 58J45,53C25,83C05.}
%\Footnotetext{}{\emph{Key words and phrases.} Nonlinear Stability, General Relativity, Einstein metrics, Einstein flow.}

\vspace{1mm}
%\affiliation{Universität Potsdam, Institut für Mathematik\\Am Neuen Palais 10\\14469 Potsdam, Germany} 
%\email{klaus.kroencke@uni-potsdam.de} 
\end{center}

%\begin{center}
%\today
%\end{center}
%
\vspace{2mm}

 
%
%\begin{center}
%\line(1,0){250}
%\end{center}
% \myrule[double]{}{}
%
%\tableofcontents
%

We thank the referee for his/her valuable the time taken
for the careful reading our paper and for the insightful comments. We think that the article has significantly improved
with the received recommendations. We have addressed the comments of the referee and we provide a list of the changes for so that
they can be more easily located in the revised version the paper which we adjoint.


  \begin{itemize}

  \item[i)] About Comment 1. Indeed we agree with the referee that the
    fact that $\xi_{AA'}$ is not a Killing vector is a conceptual
    important point to stress and explained in the text as early as
    possible. To do so, we have added what constitutes Remark 1 in
    page 6 of the revised version of the paper.

  
     
  \item[ii)] About Comment 2: We coincide with the referee that at
    first sight equations (19) and (45) do not imply that $B_{ABC}$
    and $B_{ABCD}$ vanish on the conformal boundary. We have followed
    the suggestion of the referee to clarify this point. We thank the
    referee for this comment and for suggesting a solution as
    well. The corresponding change is located in ............... in
    the revised version of the paper.

  \item[iii)] About Comment 3: We thank the referee for pointing out
    this fact about the 2-dimensional Sen connection. Although we are
    not experts on the 1+1+2 spinor split from which the 2-dimensional
    Sen connection arises, we have added reference [27] (New spinorial
    approach to mass inequalities for black holes in general
    relativity) where this is explained and could be useful for
    readers interested in further investigations of these conditions
    where the 1+1+2 split has applicability. Touching upon this point
    we have added remark 12. We acknowledge that the observation by the
    referee about the 2-dim Sen connection is of potential relevance
    for future applications of the present paper: possibly in the
    characteristic set up and for applications related to trapped
    surfaces and distorted horizons. Nevertheless, we leave this
    discussion for future work.

    
  \end{itemize}






\end{document}


